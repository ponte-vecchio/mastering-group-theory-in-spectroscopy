\section{Similarity Transformations}

The operations $\mathbf X$ and $\mathbf Y$ are said to be \textit{conjugated} if they are related by \textit{similarity transformation}, i.e. if $\textbf Z^{-1} \textbf{XZ} = \textbf{Y}$, where $\textbf Z$ is at least one operation of the group.
A \keyword{class} is a complete set of operations which are conjugate to one another.
The operations of a class have a “similar” effect and are therefore treated together.
To determine which operations of the group are in the same class as \symop[3]C, one must determine which operations are conjugate to \symop[3]C.
The results of the similarity transformation of \symop[3]C with every other member of the group are determined from the multiplication table above to be:

Thus, \symop[3]C and $\symop[3]C^2$ are conjugate and are members of the same class. In a similar manner, it can be shown that $σ_v$, $σ_v'$, and $σ_v''$ are also conjugate and these three operations form another class of the $C_{3v}$ point group.
The $C_{3v}$ point group is then written as $\left\{C_{3v}\,\middle|\,E, 2C_3, 3σ_v\right\}$. The order of a class must be an integral divisor of the order of the group.
Similar considerations allow us to write the D4h point group as $\left\{D_{4h}\,\middle|\,E, 2C_4, C_2, 2C_2', 2C_2'', i, 2S_4, σ_h, 2σ_v, 2σ_d\right\}$.

\begin{problem}
Identify the identity operator and the inverse of each function, and determine the classes for a group with the following multiplication table.
\tcblower
\begingroup
\baselineskip=1.25em
$$\openup1\jot\tabskip=0pt plus1fil
\halign to \displaywidth{\tabskip1em
    \hss$\mathbf{#}$\hss&
    \hss$#$\hss&
    \hss$#$\hss&
    \hss$#$\hss&
    \hss$#$\hss&
    \hss$#$\hss&
    \hss$#$\hss\tabskip=0pt plus1fil\cr
    \ &\mathbf M&\mathbf N&\mathbf P&\mathbf Q&\mathbf R&\mathbf S\cr\noalign{\vskip-.5\baselineskip}
    \multispan7\hrulefill\cr
    M&P&S&Q&M&N&R\cr
    N&R&Q&S&N&M&P\cr
    P&Q&R&M&P&S&N\cr
    Q&M&N&P&Q&R&S\cr
    R&S&P&N&R&Q&M\cr\noalign{\vskip-.5\baselineskip}
    \multispan7\hrulefill\cr
}$$\endgroup
\end{problem}

\pagebreak