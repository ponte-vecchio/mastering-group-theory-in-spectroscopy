\section{Direct products}

In order to determine the symmetry of the product of two \irrep{}s, it is often necessary to determine their \keyword{direct product}\textbf{s}.
What is a direct product?
It is the product of the \emph{characters} of the two representations.
For example, let us verify that under \PG{C}{2v} symmetry $\mathup A_2 \otimes \mathup B_1 = \mathup B_2$.
\begin{equation*}
    \begin{NiceArray}{ c *{4}{r} }[hvlines]
        \PG{C}{2v} & \symop E & \symop[2]C & \symop[v]{\sigma} & \symop[v]{\sigma'}\\
        \mathup A_2 & 1 & 1 & -1 & -1\\
        \mathup B_1 & 1 & -1 & 1 & -1\\
        \mathup A_2 \otimes \mathup B_1 & 1 & -1 & -1 & 1\\
    \end{NiceArray}
\end{equation*}
Just with a glimpse we can elucidate that the characters of $\mathup A_2 \otimes \mathup B_1$ are the same as those of $\mathup B_2$ \irrep.
We can, in fact, further verify that $\mathup A_2 \otimes \mathup B_2 = \mathup B_1$, and $\mathup B_1 \otimes \mathup B_2 = \mathup A_2$ and so on.
If we apply the concepts of group theory, then we can suppose a pattern:
\begin{enumerate}
    \item the product of any non-degenerate representation with itself is totally symmetric; and
    \item the product of any representation with the totally symmetric representation affords the original representation itself.
\end{enumerate}
That is to say:
\begin{align*}
    A \otimes B &= B &A \otimes A &= B \otimes B = A\\
    1 \otimes 2 &= 2 &1 \otimes 1 &= 2 \otimes 2 = 1\\
    g \otimes u &= u &g \otimes g &= u \otimes u = g
\end{align*}
The basis of selection rules is that the transition between two states, $A$ and $B$ is \emph{electric dipole allowed} if the electric dipole moment matrix element is non-zero, that is, \[\Braket{ A | \mu | B} = \int_{\infty} \mathop{\psi_{A}^*} \mu \mathop{\psi_{B}} \mathrm{d}\tau \neq 0 \] where $\mu = \mu_x + \mu_y + \mu_z$ is the \keyword{electric dipole moment operator} which transforms in the same manner as \porb{}s.
A necessary condition for this inequality is that the direct product of the integrand---$\psi_A \otimes \mu \otimes \psi_B = \psi_A \otimes \left(\mu_x + \mu_y + \mu_z\right) \otimes \psi_B$---must contain the totally symmetric representation.

Is the orbital transition $d_{yz} \to p_x$ electric dipole allowed in \PG{C}{2v} symmetry?
The \PG{C}{2v} character table indicates that the \dorb{yz} forms the basis for the $\mathup B_2$ \irrep\ while the \porb{x} transforms as a $\mathup B_1$.
Then, the question becomes whether or not a transition between a $\mathup B_2$ and a $\mathup B_1$ orbital is electric dipole allowed.
Such a transition is allowed only if the product $\mathup B_1 \otimes \mu \otimes \mathup B_2$ contains the totally symmetric representation $\mathup A_1$.
In \PG{C}{2v}, the electric dipole transforms as $\mathup b_1 + \mathup b_2 + \mathup a_1$.
The direct products are determined to be
\[
    \mathup B_1 \otimes \begin{pNiceMatrix}
        \mathup b_1 \\ \mathup b_2 \\ \mathup a_1
    \end{pNiceMatrix} \otimes \mathup B_2 = \begin{pNiceMatrix}
        \mathup a_1 \\ \mathup a_2 \\ \mathup b_1
    \end{pNiceMatrix} \otimes \mathup B_2 = \begin{pNiceMatrix}
        \mathup b_2 \\ \mathup b_1 \\ \mathup a_2
    \end{pNiceMatrix}.
\]
That is, this transition is \emph{forbidden} on the basis that none of the three components contains the $\mathup a_1$ representation.

Since the \irrep{}s are orthogonal, the direct product of two different \irrep{}s will always contain $-1$’s and thus cannot be totally symmetric.
The only way to exclusively get $+1$’s, then, is to square the individual characters i.e. the \textbf{direct product of two non-degenerate \irrep{}s can be $\mathup a_1$ only if they are the of the same \irrep.}
A triple product will transform as $\mathup A_1$ only if the direct product of two \irrep{}s is the same \irrep\ as the third.
Thus we state the following:

% box
\begin{tcolorbox}
A transition between two non-degenerate states will be allowed only if the direct product of the two state symmetries is the same \irrep\ as one of the components of the electric dipole.
\end{tcolorbox}

The transition $d_{yz} \to p_x$ wil be allowed only if the direct product $\mathup B_1 \otimes \mathup B_2$ transforms the same as $x$, $y$ or $z$.
Since $\mathup B_1 \otimes \mathup B_2 = \mathup a_2$ and $x$, $y$ nor $z$ transform as $\mathup a_2$, the transition is forbidden.
An $\mathup A_1 \to \mathup B_2$ transition is allowed, however, since $\mathup A_1 \otimes \mathup B_2 = \mathup B_2$ $y$ transforms as $b_2$.
In this case, the transition is said to be “$y$-allowed”.

\begin{problem}
    Indicate whether each of the following metal localised transitions is electric dipole allowed in \ptcl.
    \begin{enumerate}
        \item $d_{xy} \to p_z$
        \item $d_{yz} \to d_{z^2}$
        \item $d_{x^2 - y^2} \to p_x,\thinspace p_y$
        \item $p_z \to s$
    \end{enumerate}
\end{problem}