\section{The Point Groups}

The following groups all contain the identity operation and only the minimum operations required to define the group are given.
In many cases, these minimum operations lead to other operations.
In the following, $k$ is an integer $≥2$.

\begingroup\baselineskip=1em
\begin{itemize}
    \item[\PG{C}{1}] No symmetry.
    \item[\PG{C}{s}] Only a plane of symmetry.
    \item[\PG{C}{k}] Only a \symop[k]C rotational axis.
    \item[\PG{C}{i}] Only a centre of inversion.
    \item[\PG{C}{kh}] A \symop[k]C rotation axis and a \symop[h]{\sigma}.
    \item[\PG{C}{kv}] A \symop[k]C rotation axis and a \symop[v]{\sigma}.
    \item[\PG{D}{k}] One \symop[k]C and $k$\symop[2]C axes. The $k$\symop[2]C axes are orthogonal to the \symop[k]C axis and at equal angles to one another.
    \item[\PG{D}{kh}] The \symop[k]D operations plus a \symop[h]{\sigma} but this combination also results in $k\symop[v]{\sigma}$’s.
    \item[\PG{D}{kd}] The \symop[k]D operations plus a $k$\symop[d]{\sigma} containing the \symop[k]C and bisecting the angles between adjacent \symop[2]C’s.
    \item[\PG{S}{k}] Only the improper rotation \symop[k]S. Note $k$ must be an \emph{even} number since an aodd number would require a \symop[h]{\sigma}.
    \item[\PG{T}{d}] The tetrahedral point group contains three mutually perpendicular \symop[2]C axes, four \symop[3]C axes and a centre of symmetry.
    \item[\PG{O}{h}] The octahedral point group has three mutually perpendicular \symop[4]C axes, four \symop[3]C axes and a centre of symmetry.
\end{itemize}
\endgroup

Determining the point group of a molecule is the first step in a treatment of the molecular orbitals or spectra of a compound.
It is therefore important that this be done somewhat systematically.
The flow chart in \fref{fig:flowchart} is offered as an aid, and a few examples should clarify the process.
We will first determine the point groups for the following \ce{Pt(II)} ions:

\begin{figure}[!htbp]
    \centering
    \schemestart[][south]
    \chemname{\chemleft[\chemfig{%
        Pt(-[1]\clclr{5}?(-[::45,,1,1,dotted]C_2)(-[-2,,1,1,dotted]))(-[::0,,1,1,dotted])(-[4,1.1,1,1,dotted])(-[-3]\clclr{5})(-[3]\clclr{5})-[::-45]\clclr{5}(-[::135,,1,1,dotted])(-[-2,,1,1,dotted])-[::90]Pt?(-[::-90]\clclr{5})(-[::0]\clclr{5})(-[0,1.1,1,1,dotted] C_2)(-[4,,1,1,dotted])}\chemright]%
    }{\bfseries A}
    \arrow{0}
    \chemname{\chemleft[\chemfig{%
        Pt(-[1]\clclr{5}?(-[::45,,1,1,dotted]C_2)(-[-2,,1,1,dotted]))(-[-3]\clr{Br}{1})(-[3]\clclr{5})-[::-45]\clclr{5}(-[::135,,1,1,dotted])(-[-2,,1,1,dotted])-[::90]Pt?(-[::-90]\clr{Br}{1})(-[::0]\clclr{5})}\chemright]%
    }{\bfseries B}
    \arrow{0}
    \chemname{\chemleft[\chemfig{%
        Pt(-[1]\clclr{5}?)(-[-3]\clr{Br}{1})(-[3]\clclr{5})-[::-45]\clclr{5}-[::90]Pt?(-[::-90]\clclr{5})(-[::0]\clr{Br}{1})}\chemright]%
    }{\bfseries C}
    \schemestop
\end{figure}

\textbf{A} contains three \symop[2]C axes, such that $\left[C_k?\right]$ is true for $k=2$.
It contains a plane of symmetry, such that $\left[\sigma?\right]$ is also true.
The three \symop[2]C axes are perpendicular, i.e. there is a \symop[2]C axis and two perpendicular $\symop[2]C'$'s which implies that $\left[\perp C_2?\right]$ is true.
Fianlly, there is a plane of symmetry perpendicular to the \symop[2]C, thus $\left[\perp\sigma?\right]$ is also true.
Combining these four observations, we arrive at the point group \PG{D}{2h}.
\textbf{B} contains only one \symop[2]C axis, no $\perp\symop[2]C$'s, no \symop[h]{\sigma}, but it does have two $\symop[v]{\sigma}$'s---making it a \PG{C}{2v} ion.
\textbf{C} contains a single \symop[2]C axis, and a horizontal plane (i.e. the plain of the ion) therefore has \PG{C}{2h} symmetry.

% Flowchart config
\usetikzlibrary{shapes, arrows, positioning, calc}
% We need a long box for header, circle for final point group, diamond for question
\tikzstyle{header} = [
    rectangle,
    text width=0.35\textwidth,
    minimum height=1em,
    text centered,
    draw=cmap2,
    fill=white!85!cmap2
]
% diamond but stretched
\tikzstyle{spg} = [
    ellipse,
    text width=0.2\textwidth,
    text centered,
    minimum height=1em,
    x radius = 3em,
    y radius = 2.5em,
    draw=cmap1, fill=white!85!cmap1,
    rounded corners,
]
\tikzstyle{pgrp} = [
    circle,
    text centered,
    draw=cmap1,
    fill=white!85!cmap1
]
\tikzstyle{question} = [
    diamond,
    minimum width=3.3em,
    minimum height=3.3em,
    text centered,
    draw=cmap4, 
    fill=white!85!cmap4
]
\tikzstyle{arrow} = [->, >=stealth, thick]
\def\childxdist{5.4em}
\def\childydist{1.8em}

\begin{figure}[!hbtp]
    \centering
    \begin{tikzpicture}
        \node (header) [header] {Special Point Groups?\\ \small (Linear, Tetrahedral,\\ Octahedral, Icosahedral)};
        \node (special) [spg, right of = header, xshift = 3 * \childxdist] {\PG{C}{\infty v}, \PG{D}{\infty v}, \PG{T}{}, \PG{T}{d},\\ \PG{T}{h}, \PG{O}{}, \PG{O}{h}, \PG{I}{h}};
        \draw[arrow] (header) -- node[midway,above] {Yes} (special);

        \node (ck) [question, below of = header, yshift = -1.5 * \childydist] {\PG{C}{k}?};
        \draw[arrow] (header) -- node[midway,left] {No} (ck);
            \node (sigmaN) [question, right of = ck, xshift = \childxdist] {\PG{\sigma}{?}}; \draw[arrow] (ck) -- node[midway,above] {No} (sigmaN);
                \node (cs) [pgrp, right of = sigmaN, xshift = \childxdist, yshift = \childydist] {\PG{C}{s}}; \draw[arrow] (sigmaN) -- node[midway,above] {Yes} (cs);
                \node (inv) [question, right of = sigmaN, xshift = \childxdist, yshift = -\childydist] {\PG{i}{}?}; \draw[arrow] (sigmaN) -- node[midway,below] {No} (inv);
                    \node (ci) [pgrp, right of = inv, xshift = \childxdist, yshift = \childydist] {\PG{C}{i}}; \draw[arrow] (inv) -- node[midway,above] {Yes} (ci);
                    \node (cone) [pgrp, right of = inv, xshift = \childxdist, yshift = -\childydist] {\PG{C}{1}}; \draw[arrow] (inv) -- node[midway,below] {No} (cone);

            \node (sigmaY) [question, below of = ck, yshift = -1.5 * \childydist] {\PG{\sigma}{}?}; \draw[arrow] (ck) -- node[midway,left] {Yes} (sigmaY);
                \node (stwok) [question, right of = sigmaY, xshift = \childxdist] {\PG{S}{2k}?}; \draw[arrow] (sigmaY) -- node[midway,above]{Yes} (stwok);
                    \node (stwokY) [pgrp, right of = stwok, xshift = \childxdist] {\PG{S}{2k}}; \draw[arrow] (stwok) -- node[midway,above] {Yes} (stwokY);
            
            \node (Tc2) [question, below of = sigmaY, yshift = -\childydist, xshift = 0.8 * \childxdist] {$\perp \PG{C}{2}?$};
            \draw[arrow] (sigmaY) -- node[midway,left] {Yes} (Tc2);
            \draw[arrow] (stwok) -- node[midway,right] {No} (Tc2);

            \node (sigmahY) [question, below of = Tc2, xshift = -0.8 * \childxdist, yshift = -\childydist] {\PG{\sigma}{h}?}; \draw[arrow] (Tc2) -- node[midway,left] {Yes} (sigmahY);
                \node (dkh) [pgrp, left of = sigmahY, xshift = -0.666 * \childxdist] {\PG{D}{kh}}; \draw[arrow] (sigmahY) -- node[midway,above] {Yes} (dkh);
                \node (sigmavY) [question, below of = sigmahY, yshift = -1.5 * \childydist] {\PG{\sigma}{v}?}; \draw[arrow] (sigmahY) -- node[midway, right] {No} (sigmavY);
                    \node (dkd) [pgrp, left of = sigmavY, xshift = -0.666 * \childxdist] {\PG{D}{kd}}; \draw[arrow] (sigmavY) -- node[midway,above] {Yes} (dkd);
                    \node (dk) [pgrp, below of = sigmavY, yshift = -1.25 * \childydist] {\PG{D}{k}}; \draw[arrow] (sigmavY) -- node[midway,right] {No} (dk);
            \node (sigmahN) [question, below of = Tc2, xshift = 0.8 * \childxdist, yshift = -\childydist] {\PG{\sigma}{h}?}; \draw[arrow] (Tc2) -- node[midway,right] {No} (sigmahN);
                \node (ckh) [pgrp, right of = sigmahN, xshift = 0.666 * \childxdist] {\PG{C}{kh}}; \draw[arrow] (sigmahN) -- node[midway,above] {Yes} (ckh);
                \node (sigmavN) [question, below of = sigmahN, yshift = -1.5 * \childydist] {\PG{\sigma}{v}?}; \draw[arrow] (sigmahN) -- node[midway, right] {No} (sigmavN);
                    \node (ckv) [pgrp, right of = sigmavN, xshift = 0.666 * \childxdist] {\PG{C}{kv}}; \draw[arrow] (sigmavN) -- node[midway,above] {Yes} (ckv);
                    \node (ck) [pgrp, below of = sigmavN, yshift = -1.25 * \childydist] {\PG{C}{k}}; \draw[arrow] (sigmavN) -- node[midway,right] {No} (ck);   
    \end{tikzpicture}
    \caption{Flowchart for the determination of molecular point groups}\label{fig:flowchart}
\end{figure}
\pagebreak

Next, we determine the point group to which the \keyword{staggered} and \keyword{eclipsed} firms of octa\-chloro\-di\-rhenate belong.

\begin{figure}[!htbp]
    \centering
    \schemestart[][center]
    \chemname{\chemfig[cram width = .35em, cram dash width = 0.8pt, cram dash sep = 3pt]{%
        Re(<:[1,1.5,,,white!75!black]\clf{Cl}{cmap5!50!white})(<[3,1.5]\clf{Cl}{cmap5!50!white})(<[5,1.5]\clf{Cl}{cmap5!50!white})(<:[7,1.5,,,white!75!black]\clf{Cl}{cmap5!50!white})-[::-15,2,,,white!75!black]%
        Re(-[::105,1.5,,,white!40!black]\clclf3)(-[::-75,1.5,,,white!40!black]\clclf3)(<[::-155,1.5]\clclf3)(<:[::20,1.5]\clclf3)}}{\bfseries Staggered}
    \arrow{0}
    \chemname{\chemfig[cram width = .35em, cram dash width = 0.8pt, cram dash sep = 3pt]{%
    Re(<:[1,1.5]\clf{Cl}{cmap5!50!white})(<[3,1.5]\clf{Cl}{cmap5!50!white})(<[5,1.5]\clf{Cl}{cmap5!50!white})(<:[7,1.5,,,white!75!black]\clf{Cl}{cmap5!50!white})-[::-15,2,,,white!75!black]%
    Re(<:[1,1.5]\clclf3)(<[3,1.5]\clclf3)(<[5,1.5]\clclf3)(<:[7,1.5,,,white!75!black]\clclf3)}}{\bfseries Eclipsed}
    \schemestop
\end{figure}


Both forms of \recl\ contain a \symop[4]C axis (\ce{Re-Re} bond) hence the answer to $\left[\PG{C}{k}?\right]$ is true for $k = 4$.
Both also contain planes of symmetry, thus $\left[\sigma ?\right]$ is also true.
The four $\perp \symop[2]C$'s passing through the centre of the \ce{Re-Re} bond can be observed as well.
This is more apparent for the eclipsed form---2 parallel to the \ce{Re-Cl} bonds and 2 parallel to their bisectors---unlike the staggered form whose $\perp \symop[2]C$'s bisecting the \ce{Cl-Re-Re-Cl} dihedral angles are shown as below:

\begin{center}
    \schemestart[][]
        \chemfig[cram width = .35em, cram dash width = 0.8pt, cram dash sep = 3pt]{%
        Re%
        % positive (frontal) bonds
        (<[0,2]\clclf3)(<[2,2]\clclf3)(<[4,2]\clclf3)(<[6,2]\clclf3)%
        % negative bonds
        (<:[1,2,,,white!40!black]\clf{Cl}{cmap5!50!white})(<:[3,2,,,white!40!black]\clf{Cl}{cmap5!50!white})(<:[5,2,,,white!40!black]\clf{Cl}{cmap5!50!white})(<:[7,2,,,white!40!black]\clf{Cl}{cmap5!50!white})%
        % C2 axes
        (-[0.5,4,,,dotted, cmap1!50!black]\clr{\symop[2]C}{1!50!black})%
        (-[4.5,4,,,dotted, cmap1!50!black])%
        (-[-0.5,4,,,dotted, cmap2!50!black]\clr{\symop[2]C}{2!50!black})%
        (-[3.5,4,,,dotted, cmap2!50!black])%
        (-[1.5,4,,,dotted, cmap4!50!black]\clr{\symop[2]C}{4!50!black})%
        (-[-2.5,4,,,dotted, cmap4!50!black])%
        (-[-1.5,4,,,dotted, cmap5!50!black]\clr{\symop[2]C}{5!50!black})%
        (-[2.5,4,,,dotted, cmap5!50!black])%
        }
    \schemestop
\end{center}

Both forms also contain vertical planes, but the eclipsed form also has a horizontal plane which is not present in the staggered form.
The point groups are, therefore, \PG{D}{4h} for the eclipsed form and \PG{D}{4d} for the staggered form.
It is crucial to be proficient with this process, and only \emph{practice makes perfect}.
\pagebreak

\begin{problem}
Determine the point group to which each of the following belongs:
\tcblower
\vbox{\tabskip=1em\offinterlineskip
    \ialign{\tabskip=1.2em
        \hfil\schemestart[][center]#\schemestop\hfil&
        \hfil\schemestart[][center]#\schemestop\hfil&
        \hfil\schemestart[][center]#\schemestop\hfil&
        \hfil\schemestart[][center]#\schemestop\hfil&
        \hfil\schemestart[][center]#\schemestop\hfil&
        \hfil\schemestart[][center]#\schemestop\hfil\cr
        \chemfig{C(-[::180]H)~N}&
        \chemfig[cram width = .35em, cram dash width = 0.8pt, cram dash sep = 3pt]{Sn(-[3,1.3]Cl)(-[-3,1.3]Cl)(<[::-20,1.3]Br)(<:[::20,1.3]Br)}&
        \chemfig[cram width = .35em, cram dash width = 0.8pt, cram dash sep = 3pt]{Sn(-[3,1.3]Cl)(-[-3,1.3]Cl)(<[::-20,1.3]Br)(<:[::20,1.3]Cl)}&
        \chemfig[cram width = .3em, cram dash width = 0.8pt, cram dash sep = 3pt]{?<[1.333,1.1]-[::-120,1.1,,,line width=.3em]>[::120,1.1]-[-2,0.8]-[::-150]-[::120]-[::-120]?}&
        \chemfig{Xe(-F)(-[::180]F)}&
        \chemfig[cram width = .35em, cram dash width = 0.8pt, cram dash sep = 3pt]{W(=[2]Se)(=[-2]Se)(<[::-20,1.3]P)(<:[::20,1.3]P)(<[::160,1.3]P)(<:[::-160,1.3]P)}\cr\noalign{\bigskip}
        \chemfig[cram width = .35em, cram dash width = 0.8pt, cram dash sep = 3pt]{Pt(<[::-20,1.3]Cl)(<[::-160,1.3]Br)(<:[::20,1.3]Cl)(<:[::160,1.3]Br)(-[2]Cl)(-[-2]Br)}&
        \chemfig[cram width = .35em, cram dash width = 0.8pt, cram dash sep = 3pt]{Pt(<[::-20,1.3]Cl)(<[::-160,1.3]Br)(<:[::20,1.3]Br)(<:[::160,1.3]Br)(-[2]Cl)(-[-2]Cl)}&
        \chemfig[cram width = .35em, cram dash width = 0.8pt, cram dash sep = 3pt]{Pt(<[::-20,1.3]Cl)(<[::-160,1.3]Br)(<:[::20,1.3]Cl)(<:[::160,1.3]Br)(-[2]Cl)(-[-2]Cl)}&
        \chemfig[cram width = .35em, cram dash width = 0.8pt, cram dash sep = 3pt]{Re(<[::-20,1.3]CN)(<[::-160,1.3,,2]NC)(<:[::20,1.3]CN)(<:[::160,1.3,,2]NC)(=[2]O)(=[-2]O)}&
        \chemfig[cram width = .35em, cram dash width = 0.8pt, cram dash sep = 3pt]{V(<[::-20]@{od}O)(<[::-160]@{oa}O)(<:[::20]@{oc}O)(<:[::160]@{ob}O)(=[2]O)(-[-2]py)}%
        \chemmove{\draw[-](oa)..controls ++(170:1em) and ++(-170:1em)..(ob);}%
        \chemmove{\draw[-](oc)..controls ++(-10:1em) and ++(10:1em)..(od);}&
        \chemfig[cram width = .35em, cram dash width = 0.8pt, cram dash sep = 3pt]{V(<[::-20]@{ob}O)(<[::-160]@{oc}O)(<:[::20]@{oa}O)(<:[::160]py)(=[2]O)(-[-2]@{od}O)}%
        \chemmove{\draw[-](oa)..controls ++(-10:1em) and ++(10:1em)..(ob);}%
        \chemmove{\draw[-](od)..controls ++(-140:1.25em) and ++(-110:1.25em)..(oc);}
        \cr\noalign{\bigskip}
        \chemfig[cram width = .35em, cram dash width = 0.8pt, cram dash sep = 3pt]{%
            Ru(<[::-20,1.3]CN)(<:[::20,1.3]CN)%
            (-[2]N?[na])(<:[::160,1.3]N-[2,0.7]-[::-65,0.7]?[na])%
            (-[-2]N?[nb])(<[::-160,1.3]N-[-2,0.65]-[::65,0.7]?[nb])%
        }&
        \chemfig[cram width = .35em, cram dash width = 0.8pt, cram dash sep = 3pt]{Mo(-[::90]N)(<:[::-15,1.3]Cl)(<[::-55,1.3]O)(<[::-120,1.3]Cl)(<:[::-165,1.3]O)}&
        \chemfig[cram width = .35em, cram dash width = 0.8pt, cram dash sep = 3pt]{Cu(-[3,1.3]I)(-[-3,1.3]I)(<[::-20,1.3]Br)(<:[::20,1.3]Cl)}&
        \chemfig[cram width = .35em, cram dash width = 0.8pt, cram dash sep = 3pt]{%
            (-[::0,0.8](-[::90]Cl)(-[::-90]Cl)(<[::-20]F)(<:[::160]F))%
            (-[::180,0.8](<[::120]F)(<[::-120]Cl)(<:[::60]Cl)(<:[::-60]F))%
        }&
        \chemfig[cram width = .35em, cram dash width = 0.8pt, cram dash sep = 3pt]{%
            (-[::90]F)(-[0,1.2]F)(-[::-90]F)(<:[::160,1.2]F)(<[::-160,1.2]Cl)
        }&
        \chemfig{Sb(=[0,1.1]O)-[::-150,1.1]Cl}\cr
    }
}
\end{problem}