All of the topics covered in this course make extensive use of molecular symmetry and group theory which are, by themselves, topics of entire books.
It is the aim of this text to present group theory in sufficient detail so that the student can solve most day-to-day spectroscopic problems and read the literature with confidence.

A \keyword{symmetry operation} is a rotation and/or reflection which leaves the molecule unchanged. 
It is performed about a \keyword{symmetry element}---a point, a line or a plane.
The square planar \ptcl\ ion is reported to be a highly symmetric ion because it contains a large number of symmetry elements.
All of the symmetry operations that apply to a molecule or ion constitute its \textit{point group}.
It is the point group to which the molecule belongs that designates its symmetry.
The following section will exhibit sixteen symmetry operations that can be performed on the \ptcl\ ion. 
These operations constitute what is known as the \PG{D}{4h} point group, i.e. \ptcl\ has \PG{D}{4h} symmetry.
In a treatment of the spectroscopy or bondings of an inorganic compound, one must first determine the point group to which the system belongs.
A full appreciation of what \PG{D}{4h} symmetry implies requires the visualisation of the symmetry elements and operations as well as an understanding of groups.
In the following two sections thereafter, various symmetry operations as well as groups will be discussed.
The remainder of the chapter is devoted to representations of the point groups and to some simple applications of group theory to spectroscopy and bonding.

\section{Symmetry operation and symmetry elements}

There are only 5 types of symmetry operations required for the systems covered in typical undergraduate curricula, and \ptcl\  ion contains examples of each.
In this section, each type of operation will be discussed in terms of its effect on the ion.
The effect of the operations on the chlorine \porb{z}s will also be considered in order to better illustrate the effects of the operations.
Thus, in the figures that follow, \clclr1, \clclr2, \clclr3\ and \clclr4\ will represent \ce{Cl} atom 1 with the positive lobe of the \porb{z} out of the plane of the paper, while \clfinline1, \clfinline2, \clfinline3\ and \clfinline4\ will imply that the negative lobe of the \porb{z} is out of the plane of the paper for \ce{Cl} atom 1.


1. The \keyword{Identity Operation}, (\symop{E}) does nothing and has no \keyword{symmetry element} but it is a required member of each symmetry group. Thus, operation with \symop{E} will change neither the positions of the atoms nor the phaes of the \porb{z}s.

\begin{figure}[htbp!]
    \centering
    \begin{tabular}{r | l}
        \omit\hss Atoms\hss & \porb{z}s\\
        \schemestart[][north]
        \chemfig{Pt(-[2]\clclr2)(-[4]\clclr1)(-[6]\clclr4)(-[0]\clclr3)}
        \arrow{->[\symop{E}]}
        \chemfig{Pt(-[2]\clclr2)(-[4]\clclr1)(-[6]\clclr4)(-[0]\clclr3)}
        \schemestop&
        \schemestart[][north]
        \chemfig{Pt(-[2]\clclr2)(-[4]\clclr1)(-[6]\clclr4-[::0,,,,dotted]Y)(-[0]\clclr3-[::0,,,,dotted]X)(-[1,1.5,,,dotted]a)(-[5,1.5,,,dotted])(-[3,1.5,,,dotted])(-[7,1.5,,,dotted]b)}
        \schemestop
    \end{tabular}
    \caption{The effect of the identity operation on the atoms and the \ce{Cl} $p_z$ orbitals in \ptcl} \label{fig:identity-pz}
\end{figure}

2. An \keyword{N-fold rotation}, (\symop[n]{C}) is a rotation of $\frac{2\pi}{n}$ radians about an axis.
The axis with the \textbf{highest value of $\mathbf n$ is the} \keyword{principle axis}, and is designated as the $z$-axis.
Thus the $z$-axis in \ptcl\ is perpendicular to the plane of the ion.
This axis is in fact three symmetry elements since rotations by $\frac{\pi}{2}$, $\pi$ and $\frac{3\pi}{2}$ about this axis all yield no change in the molecule.
These three axes are referred to as $\symop[4]{C},\,\symop[4]{C}^2,\,\symop[2]{C}$ and $\symop[4]{C}^3$ respectively.\footnotemark\ 
Rotations about the $z$-axis will not change the phase of the \porb{z}s.
The $X$, $Y$, $a$ and $b$ axes defined in \fref{fig:identity-pz} are also \symop[2]{C} rotational axes.
It will be shown later that the \symop[2]{C} rotations in this group can be grouped into three \keyword{classes} which are differentiated with the use of $'\/$ and $''\/$\ $\left\{\symop[2]{C}\left(Z\right)\right\}$, $\left\{\symop[2]{C}'\left(X\right)\,\&\,\symop[2]{C}'\left(Y\right)\right\}$, and $\left\{\symop[2]{C}''\left(a\right)\,\&\,\symop[2]{C}''\left(b\right)\right\}$.
Since the $\symop[2]{C}'$ and $\symop[2]{C}''$ are orthogonal to the $z$-axis, rotation about any one of them will invert the \porb{z}s as shown in \fref{fig:rotation-pz}.

\begin{figure}[!htbp]
    \centering 
    \begin{tabular}{r | l}
        \omit\hss Atoms\hss & \porb{z}s\\
        \schemestart[][north]
        \chemfig{Pt(-[2]\clclr2)(-[4]\clclr1)(-[6]\clclr4)(-[0]\clclr3)}
        \arrow{->[$\symop[4]{C}\left(Z\right)$]}
        \chemfig{Pt(-[2]\clclr1)(-[4]\clclr4)(-[6]\clclr3)(-[0]\clclr2)}
        \schemestop&
        \schemestart[][north]
        \chemfig{Pt(-[2]\clclr1)(-[4]\clclr4)(-[6]\clclr3)(-[0]\clclr2)}
        \schemestop\\
        \schemestart[][north]
        \chemfig{Pt(-[2]\clclr2)(-[4]\clclr1)(-[6]\clclr4)(-[0]\clclr3)}
        \arrow{->[$\symop[4]{C}^3\left(Z\right)$]}
        \chemfig{Pt(-[2]\clclr3)(-[4]\clclr2)(-[6]\clclr1)(-[0]\clclr4)}
        \schemestop&
        \schemestart[][north]
        \chemfig{Pt(-[2]\clclr3)(-[4]\clclr2)(-[6]\clclr1)(-[0]\clclr4)}
        \schemestop\\
        \schemestart[][north]
        \chemfig{Pt(-[2]\clclr2)(-[4]\clclr1)(-[6]\clclr4)(-[0]\clclr3)}
        \arrow{->[$\symop[2]{C}'\left(X\right)$]}
        \chemfig{Pt(-[2]\clclr4)(-[4]\clclr1)(-[6]\clclr2)(-[0]\clclr3)}
        \schemestop&
        \schemestart[][north]
        \chemfig{Pt(-[2]\clclf{4})(-[4]\clclf{1})(-[6]\clclf{3})(-[0]\clclf{2}-[::0,,,,dotted]X)}
        \schemestop\\
        \schemestart[][north]
        \chemfig{Pt(-[1,1.5,,,dotted]a)(-[5,1.5,,,dotted])(-[2]\clclr2)(-[4]\clclr1)(-[6]\clclr4)(-[0]\clclr3)}
        \arrow{->[$\symop[2]{C}''\left(a\right)$]}
        \chemfig{Pt(-[2]\clclr3)(-[4]\clclr4)(-[6]\clclr1)(-[0]\clclr2)}
        \schemestop&
        \schemestart[][north]
        \chemfig{Pt(-[2]\clclf3)(-[4]\clclf4)(-[6]\clclf1)(-[0]\clclf2)}
        \schemestop\\
    \end{tabular}
    \caption{The effect of some of the \symop[4]{C} operations on the atoms and the \ce{Cl} $p_z$ orbitals in \ptcl.} \label{fig:rotation-pz}
\end{figure}

Thus the ion contains seven rotational axes, namely \symop[4]C, $\symop[4]C^3$, $\symop[4]C^2 \equiv \symop[2]C$, $\symop[2]C' (X)$, $\symop[2]C' (Y)$, $\symop[2]C'' (a)$ and $\symop[2]C'' (b)$.

\footnotetext{Since the clockwise $\symop[4]{C}^3$ operation is equivalent to a counterclockwise \symop[4]{C} rotation, the \symop[4]{C} and $\symop[4]{C}^3$ operations are also referred to as the $\symop[4]{C}^+$ and $\symop[4]{C}^-$ operations respectively.}

3. \keyword{Reflection}\textbf{s} can be made through three different types of planes: \keyword{vertical plane}\textbf{s} \(\left(\symop[v]{\sigma}\right)\) contains the principle axis; \keyword{horizontal plane}\textbf{s} \(\left(\symop[h]{\sigma}\right)\) are orthogonal to the principle axis; and \keyword{dihedral plane}\textbf{s} \(\left(\symop[d]{\sigma}\right)\) contain the principle axis and bisect two \symop[2]{C} axes.
The distinction between vertical and diheral is often unclear.
Where appropriate, planes bisecting bond angles will be designated as dihedral while those contaning bonds will be designated as vertical.
See \fref{fig:reflection-pz}.

\begin{figure}[!hbtp]
    \centering
    \begin{tabular}{ r | l }
        \omit\hss Atoms \hss & \porb{z}s\\
        \schemestart[][north]
        \chemfig{Pt(-[2]\clclr2)(-[4]\clclr1)(-[6]\clclr3)(-[0]\clclr4)}
        \arrow{->[$\symop[v]{\sigma} (yz)$]}
        \chemfig{Pt(-[2]\clclr2)(-[4]\clclr3)(-[6]\clclr4)(-[0]\clclr1)}
        \schemestop&
        \schemestart[][north]
        \chemfig{Pt(-[2]\clclr2)(-[4]\clclr3)(-[6]\clclr4)(-[0]\clclr1)}
        \schemestop\\
        \schemestart[][north]
        \chemfig{Pt(-[2]\clclr2)(-[4]\clclr1)(-[6]\clclr4)(-[0]\clclr3)(-[1,1.5,,,dotted]a)(-[5,1.5,,,dotted])}
        \arrow{->[$\symop[d]{\sigma} (a)$]}
        \chemfig{Pt(-[2]\clclr3)(-[4]\clclr4)(-[6]\clclr1)(-[0]\clclr2)}
        \schemestop&
        \schemestart[][north]
        \chemfig{Pt(-[2]\clclr3)(-[4]\clclr4)(-[6]\clclr1)(-[0]\clclr2)}
        \schemestop\\
        \schemestart[][north]
        \chemfig{Pt(-[2]\clclr2)(-[4]\clclr1)(-[6]\clclr4)(-[0]\clclr3)}
        \arrow{->[$\symop[h]{\sigma} (xy)$]}
        \chemfig{Pt(-[2]\clclr2)(-[4]\clclr1)(-[6]\clclr4)(-[0]\clclr3)}
        \schemestop&
        \schemestart[][north]
        \chemfig{Pt(-[2]\clclf2)(-[4]\clclf1)(-[6]\clclf4)(-[0]\clclf3)}
        \schemestop\\
    \end{tabular}
    \caption{The effect of reflection through the symmetry planes on the atoms and the \ce{Cl} $p_z$ orbitals in \ptcl.} \label{fig:reflection-pz}
\end{figure}

In \ptcl, the planes containing the $z$-axis (\symop[v]{\sigma} and \symop[d]{\sigma}) will not change the phase of the \porb{z}s while reflection through the plane perpendicular to the $z$-axis (\symop[h]{\sigma}) does invert them (\fref{fig:reflection-pz}).
Thus, \ptcl\ contains five planes of symmetry: namely \(\symop[v]{\sigma} (XZ)\), \(\symop[v]{\sigma} (YZ)\), \(\symop[h]{\sigma} (XY)\), \(\symop[d]{\sigma} (a)\) and \(\symop[d]{\sigma} (b)\).
Note that the $a$ and $b$ planes are defined as those planes orthogonal to the plane of the ion and contaning the $a$ and $b$ rotational axes.

4. An \keyword{improper rotation} or a \keyword{rotary reflection} (\symop[n]S) is a \symop[n]{C} followed by a \symop[h]{\sigma}.
Since \ptcl\ is a planar ion, the $Z$-axis is an element for both proper and improper rotations.
See \fref{fig:improper-rotation-pz}.
Note that an \symop[4]S result in the same numbering as a \symop[4]C, but the phases of the \porb{z}s are changed.

\begin{figure}[!htbp]
    \centering
    \begin{tabular}{r | l}
        \omit \hss Atoms \hss & \porb{z}s\\
        \schemestart[][north]
        \chemfig{Pt(-[2]\clclr2)(-[4]\clclr1)(-[6]\clclr4)(-[0]\clclr3)}
        \arrow{->[\symop[4]{S}]}
        \chemfig{Pt(-[2]\clclr1)(-[4]\clclr4)(-[6]\clclr3)(-[0]\clclr2)}
        \schemestop&
        \schemestart[][north]
        \chemfig{Pt(-[2]\clclf1)(-[4]\clclf4)(-[6]\clclf3)(-[0]\clclf2)}
        \schemestop\\
        \schemestart[][north]
        \chemfig{Pt(-[2]\clclr2)(-[4]\clclr1)(-[6]\clclr4)(-[0]\clclr3)}
        \arrow{->[\(i = \symop[2]{S}\)]}
        \chemfig{Pt(-[2]\clclr4)(-[4]\clclr3)(-[6]\clclr2)(-[0]\clclr1)}
        \schemestop&
        \schemestart[][north]
        \chemfig{Pt(-[2]\clclf4)(-[4]\clclf3)(-[6]\clclf2)(-[0]\clclf1)}
        \schemestop\\
    \end{tabular}
    \caption{The effect of improper rotations on atoms and the \ce{Cl} $p_z$ orbitals in \ptcl.}\label{fig:improper-rotation-pz}
\end{figure}

5. A \keyword{centre of inversion} \(\symop i\) takes all $(x, y, z) \to (-x, -y, -z)$. This operation can be performed by a \(\symop[2]C (z)\) which takes $(x, y, z) \to (-x, -y, +z)$ followed by a \(\symop{\sigma} (xy)\) which inverts $z$, i.e. $\symop i = \symop[2]C\symop[h]{\sigma} = \symop[2]S$. Since \symop i and \(\symop[2]S\) are equivalent, \(\symop[2]S\) is rarely used.

In sum, \ptcl\ contains \symop E, \symop[4]C, $\symop[4]C^3$, $\symop[4]C^2 = \symop[2]C$, $\symop[2]C' (x)$, $\symop[2]C' (y)$, $\symop[2]C'' (a)$, $\symop[2]C'' (b)$, \symop i, \symop[4]S, $\symop[4]S^3$, \(\symop[v]{\sigma} (XZ)\), \(\symop[v]{\sigma} (YZ)\), \(\symop[h]{\sigma}\), \(\symop[d]{\sigma} (a)\) and \(\symop[d]{\sigma} (b)\).
These 16 symmetry elements specify the symmetry of the ion.

\paragraph*{\keyword{Successive Operations}}
In some of the following discussion, the result of applying more than one operation will be of importance.
The result of performing a \symop[4]C rotation followed by a reflection through the $XZ$ plan (\symop[v]{\sigma}\symop[4]C) is the same as a single \(\symop[d]{\sigma} (a)\) operation.

\begin{center}
    \schemestart[][]
    \chemfig{Pt(-[2]\clclr2)(-[4]\clclr1)(-[6]\clclr4)(-[0]\clclr3)}
    \arrow{->[\symop[4]{C}]}
    \chemfig{Pt(-[2]\clclr1)(-[4]\clclr4)(-[6]\clclr3)(-[0]\clclr2)}
    \arrow{->[$\symop[v]{\sigma} (xz)$]}
    \chemfig{Pt(-[2]\clclr3)(-[4]\clclr4)(-[6]\clclr1)(-[0]\clclr2)}
    \arrow{->}
    $\symop[d]{\sigma} (a)$
    \schemestop
\end{center}

\noindent However, if the order of the operations is reversed, i.e. (\symop[4]C \symop[v]{\sigma}) the result is equivalent to $\symop[d]{\sigma} (b)$.

\begin{center}
    \schemestart[][]
    \chemfig{Pt(-[2]\clclr2)(-[4]\clclr1)(-[6]\clclr4)(-[0]\clclr3)}
    \arrow{->[$\symop[v]{\sigma} (xz)$]}
    \chemfig{Pt(-[2]\clclr4)(-[4]\clclr1)(-[6]\clclr2)(-[0]\clclr3)}
    \arrow{->[\symop[4]{C}]}
    \chemfig{Pt(-[2]\clclr1)(-[4]\clclr2)(-[6]\clclr3)(-[0]\clclr4)}
    \arrow{->}
    $\symop[d]{\sigma} (b)$
    \schemestop
\end{center}

It is important to note that in this case the order of the operation is important, i.e. $\symop[4]{C}\symop[v]{\sigma} ≠ \symop[v]{\sigma}\symop[4]{C}$. In some instances, the order of operation is not important. Two operations commute if the result of successive application of the two operators is the same irrespective of the order in which they were carried out. Thus, \symop[4]{C} and \symop[v]{\sigma} do not commute, but, as shown below, \symop[4]{C} and $\symop[h]{\sigma}$ do commute, i.e., $\symop[v]{\sigma} = \symop[4]S = \symop[h]{\sigma}\symop[4]C$.

\begin{center}
    \schemestart[][north]
    \chemfig{Pt(-[2]\clclr2)(-[4]\clclr1)(-[6]\clclr4)(-[0]\clclr3)}
    \arrow{->[\symop[4]{C}]}
    \chemfig{Pt(-[2]\clclr1)(-[4]\clclr4)(-[6]\clclr3)(-[0]\clclr2)}
    \arrow{->[$\symop[h]{\sigma}$]}
    \chemfig{Pt(-[2]\clclf1)(-[4]\clclf4)(-[6]\clclf3)(-[0]\clclf2)}
    \arrow{->}
    $\symop[4]S$
    \schemestop
    \vskip\baselineskip
    \schemestart[][north]
    \chemfig{Pt(-[2]\clclr2)(-[4]\clclr1)(-[6]\clclr4)(-[0]\clclr3)}
    \arrow{->[$\symop[h]{\sigma}$]}
    \chemfig{Pt(-[2]\clclf2)(-[4]\clclf1)(-[6]\clclf4)(-[0]\clclf3)}
    \arrow{->[\symop[4]{C}]}
    \chemfig{Pt(-[2]\clclf1)(-[4]\clclf4)(-[6]\clclf3)(-[0]\clclf2)}
    \arrow{->}
    $\symop[4]S$
    \schemestop
\end{center}
\pagebreak
%% End of section 1
\section{Groups}

A set of operations like those above form a \keyword{group} if they satisfy the following four conditions:

\begin{enumerate}
    \item The set contains the identity operation \symop{E} such that $\symop{RE} = \symop{ER} = \symop{R}$ \forall $\symop{R}$ in the set.
    \item The \textbf{product of any two operations of the group must also be a member of the group}. From above, it should be clear that $\symop[4]C \symop[v]{\sigma} (XZ) = \symop[d]{\sigma} (b)$ while $\symop[v]{\sigma} (XZ)\symop[4]C = \symop[d]{\sigma} (a)$ .
    \item \textbf{Multiplication is associate for all members of the group}. The triple product $\symop[2]C (a) \symop[v]{\sigma} (XZ) \symop[4]C$ can be written either as $\left\{\symop[2]C (a) \symop[v]{\sigma} (XZ)\right\}\symop[4]C = \symop[4]S^3\symop[4]C = \symop[h]{\sigma}$ or as $\symop[2]C (a) \left\{\symop[v]{\sigma} (XZ) \symop[4]C\right\} = \symop[2]C (a) \symop[d]{\sigma} (a) = \symop[h]{\sigma}$.
    \item The \textbf{inverse of every operation is a member of the group}. The inverse of an operation which returns the system to its original form, i.e. $\symop{RR^{-1}} = \symop{E}$. A reflection through a plane and a two-fold rotation are each their own inverse.
\end{enumerate}

The 16 symmetry operations discussed for the \ptcl\ ion satisfy all of these requirements and constitute a group.
Groups of symmetry operations are called \keyword{point group}\textbf s.
As mentioned previously, the point group to which \ptcl\ belongs is called \PG{D}{4h}.
The number of members of the group is called the \textbf{order} of the group and given the symbol $h$. For the \PG{D}{4h} point group, $h=16$.

A \keyword{multiplication table} presents the results of the multiplication, i.e. the successive application of two operations.
By convention, the \emph{first} operation performed is given at the \emph{top of the column} and the \emph{second} operation involved is at the \emph{beginning of the row}.
The multiplication table for the \symop{E}, \symop[4]C, \symop[2]C and $\symop[4]C^3$ operations of the \PG{D}{4h} point group is given below.

\begingroup
\renewcommand{\arraystretch}{1.35}
\renewcommand{\tabcolsep}{1em}
\begin{table}[!htbp]
    \centering
    \baselineskip=1.5em
    \begin{tabular}{ r | c c c c }
    \toprule
    First \rightarrow&
    \multirow{2}{*}{\symop{E}}&
    \multirow{2}{*}{\symop[4]C}&
    \multirow{2}{*}{\symop[2]C}&
    \multirow{2}{*}{$\symop[4]C^3$}\\
    Second \downarrow&&&&\\
    \midrule
    \symop{E}& \symop{E}& \symop[4]C& \symop[2]C& $\symop[4]C^3$\\
    \symop[4]C& \symop[4]C& \symop[2]C& $\symop[4]C^3$& \symop{E}\\
    \symop[2]C& \symop[2]C& $\symop[4]C^3$& \symop{E}& \symop[4]C\\
    $\symop[4]C^3$& $\symop[4]C^3$& \symop{E}& \symop[4]C& \symop[2]C\\
    \bottomrule
    \end{tabular}
\end{table}
\endgroup

These operations satisfy all of the requirements of a group of order $4$ ($h=4$).
Indeed, they comprise the \symop[4]C point group.
Since all of the members of the \symop[4]C point group are also found in the D4h point group ($h=16$), \symop[4]C is said to be a \keyword{subgroup} of \PG{D}{4h}.
Note that the order of a subgroup must be an integral divisor of the order of the group.

\begin{problem}
   Water belongs to the \PG{C}{2v} point group, $\left\{C_{2v}\,\middle|\, E, C_2, σ (XZ), σ (YZ)\right\}$. Define the molecular plane as the $\mathbf{XZ}$ plane and generate the multiplication table for the \PG{C}{2v} point group. 
\end{problem}

In the following section, extensive use of the multiplication table will be made, but since the point group is so large, its multiplication table is cumbersome ($16\times 16$).
We will, therefore, consider the ammonia molecule which has lower symmetry.
\ce{NH3} belongs to the \PG{C}{3v} point group of order $6$, $\left\{C_{3v}\,\middle|\,E, C_3, C_3^2, σ_v, σ_v',σ_v'' \right\}$.
The effect of each of the symmetry operations of the \PG{C}{3v} point group on the ammonia molecule is shown in \fref{fig:ammonia}.

\begin{figure}[!htbp]
    \centering
    \begingroup\renewcommand\tabcolsep{2em}
    \begin{tabular}{c c}
        \schemestart[][north]
        \chemfig{N(-[2]\clr{H}{1})(-[::-30]\clr{H}{3})(-[::210]\clr{H}{2})}
        \arrow{->[\symop{E}]}
        \chemfig{N(-[2]\clr{H}{1})(-[::-30]\clr{H}{3})(-[::210]\clr{H}{2})}
        \schemestop&
        \schemestart[][north]
        \chemfig{N(-[2]\clr{H}{1})(-[::-30]\clr{H}{3})(-[::210]\clr{H}{2})}
        \arrow{->[\symop[3]C]}
        \chemfig{N(-[2]\clr{H}{3})(-[::-30]\clr{H}{2})(-[::210]\clr{H}{1})}
        \schemestop\\
        \schemestart[][north]
        \chemfig{N(-[2]\clr{H}{1})(-[::-30]\clr{H}{3})(-[::210]\clr{H}{2})}
        \arrow{->[$\symop[3]C^2$]}
        \chemfig{N(-[2]\clr{H}{2})(-[::-30]\clr{H}{1})(-[::210]\clr{H}{3})}
        \schemestop&
        \schemestart[][north]
        \chemfig{N(-[2]\clr{H}{1})(-[::-30]\clr{H}{3})(-[::210]\clr{H}{2})(-[::-90,,,,dotted])}
        \arrow{->[$\symop[v]{\sigma}$]}
        \chemfig{N(-[2]\clr{H}{1})(-[::-30]\clr{H}{2})(-[::210]\clr{H}{3})}
        \schemestop\\
        \schemestart[][north]
        \chemfig{N(-[2]\clr{H}{1})(-[::-30]\clr{H}{3})(-[::210]\clr{H}{2})(-[::150,,,,dotted])}
        \arrow{->[$\symop[v]{\sigma}'$]}
        \chemfig{N(-[2]\clr{H}{3})(-[::-30]\clr{H}{1})(-[::210]\clr{H}{2})}
        \schemestop&
        \schemestart[][north]
        \chemfig{N(-[2]\clr{H}{1})(-[::-30]\clr{H}{3})(-[::210]\clr{H}{2})(-[::30,,,,dotted])}
        \arrow{->[$\symop[v]{\sigma}''$]}
        \chemfig{N(-[2]\clr{H}{2})(-[::-30]\clr{H}{3})(-[::210]\clr{H}{1})}
        \schemestop
    \end{tabular}
    \endgroup
    \caption{The effect of each of the symmetry operations of the $C_{3v}$ point group on the ammonia molecule as viewed down the \symop[3]C axis.}
    \label{fig:ammonia}
\end{figure}

The student should verify that the multiplication table for the $C_{3v}$ point group is therefore:
\begin{table}[!hbtp]
    \centering
    \begingroup
    \renewcommand{\arraystretch}{1.35}
    \renewcommand{\tabcolsep}{0.5em}
    \begin{tabular}{r | c c c c c c}
        \toprule
        First\rightarrow&
        \multirow{2}{*}{\symop{E}}&
        \multirow{2}{*}{$\symop[3]C$}&
        \multirow{2}{*}{$\symop[3]C^2$}&
        \multirow{2}{*}{$\symop[v]{\sigma}$}&
        \multirow{2}{*}{$\symop[v]{\sigma}'$}&
        \multirow{2}{*}{$\symop[v]{\sigma}''$}\\
        Second\downarrow&&&&&&\\
        \midrule
        \symop{E}& \symop{E}& $\symop[3]C$& $\symop[3]C^2$& $\symop[v]{\sigma}$& $\symop[v]{\sigma}'$& $\symop[v]{\sigma}''$\\
        $\symop[3]C$& $\symop[3]C$& $\symop[3]C^2$& \symop{E}& $\symop[v]{\sigma}''$& $\symop[v]{\sigma}$& $\symop[v]{\sigma}'$\\
        $\symop[3]C^2$& $\symop[3]C^2$& \symop{E}& $\symop[3]C$& $\symop[v]{\sigma}'$& $\symop[v]{\sigma}''$& $\symop[v]{\sigma}$\\
        $\symop[v]{\sigma}$& $\symop[v]{\sigma}$& $\symop[v]{\sigma}'$& $\symop[v]{\sigma}''$& \symop{E}& $\symop[3]C$& $\symop[3]C^2$\\
        $\symop[v]{\sigma}'$& $\symop[v]{\sigma}'$& $\symop[v]{\sigma}''$& $\symop[v]{\sigma}$& $\symop[3]C^2$& \symop{E}& $\symop[3]C$\\
        $\symop[v]{\sigma}''$& $\symop[v]{\sigma}''$& $\symop[v]{\sigma}$& $\symop[v]{\sigma}'$& $\symop[3]C$& $\symop[3]C^2$& \symop{E}\\
        \bottomrule
    \end{tabular}
    \endgroup
\end{table}
\pagebreak
%% End of section 2
\section{Similarity Transformations}

The operations $\mathbf X$ and $\mathbf Y$ are said to be \textit{conjugated} if they are related by \textit{similarity transformation}, i.e. if $\textbf Z^{-1} \textbf{XZ} = \textbf{Y}$, where $\textbf Z$ is at least one operation of the group.
A \keyword{class} is a complete set of operations which are conjugate to one another.
The operations of a class have a “similar” effect and are therefore treated together.
To determine which operations of the group are in the same class as \symop[3]C, one must determine which operations are conjugate to \symop[3]C.
The results of the similarity transformation of \symop[3]C with every other member of the group are determined from the multiplication table above to be:

Thus, \symop[3]C and $\symop[3]C^2$ are conjugate and are members of the same class. In a similar manner, it can be shown that $σ_v$, $σ_v'$, and $σ_v''$ are also conjugate and these three operations form another class of the $C_{3v}$ point group.
The $C_{3v}$ point group is then written as $\left\{C_{3v}\,\middle|\,E, 2C_3, 3σ_v\right\}$. The order of a class must be an integral divisor of the order of the group.
Similar considerations allow us to write the D4h point group as $\left\{D_{4h}\,\middle|\,E, 2C_4, C_2, 2C_2', 2C_2'', i, 2S_4, σ_h, 2σ_v, 2σ_d\right\}$.

\begin{problem}
Identify the identity operator and the inverse of each function, and determine the classes for a group with the following multiplication table.
\tcblower
\begingroup
\baselineskip=1.25em
$$\openup1\jot\tabskip=0pt plus1fil
\halign to \displaywidth{\tabskip1em
    \hss$\mathbf{#}$\hss&
    \hss$#$\hss&
    \hss$#$\hss&
    \hss$#$\hss&
    \hss$#$\hss&
    \hss$#$\hss&
    \hss$#$\hss\tabskip=0pt plus1fil\cr
    \ &\mathbf M&\mathbf N&\mathbf P&\mathbf Q&\mathbf R&\mathbf S\cr\noalign{\vskip-.5\baselineskip}
    \multispan7\hrulefill\cr
    M&P&S&Q&M&N&R\cr
    N&R&Q&S&N&M&P\cr
    P&Q&R&M&P&S&N\cr
    Q&M&N&P&Q&R&S\cr
    R&S&P&N&R&Q&M\cr\noalign{\vskip-.5\baselineskip}
    \multispan7\hrulefill\cr
}$$\endgroup
\end{problem}

\pagebreak
\section{The Point Groups}

The following groups all contain the identity operation and only the minimum operations required to define the group are given.
In many cases, these minimum operations lead to other operations.
In the following, $k$ is an integer $≥2$.

\begingroup\baselineskip=1em
\begin{itemize}
    \item[\PG{C}{1}] No symmetry.
    \item[\PG{C}{s}] Only a plane of symmetry.
    \item[\PG{C}{k}] Only a \symop[k]C rotational axis.
    \item[\PG{C}{i}] Only a centre of inversion.
    \item[\PG{C}{kh}] A \symop[k]C rotation axis and a \symop[h]{\sigma}.
    \item[\PG{C}{kv}] A \symop[k]C rotation axis and a \symop[v]{\sigma}.
    \item[\PG{D}{k}] One \symop[k]C and $k$\symop[2]C axes. The $k$\symop[2]C axes are orthogonal to the \symop[k]C axis and at equal angles to one another.
    \item[\PG{D}{kh}] The \symop[k]D operations plus a \symop[h]{\sigma} but this combination also results in $k\symop[v]{\sigma}$’s.
    \item[\PG{D}{kd}] The \symop[k]D operations plus a $k$\symop[d]{\sigma} containing the \symop[k]C and bisecting the angles between adjacent \symop[2]C’s.
    \item[\PG{S}{k}] Only the improper rotation \symop[k]S. Note $k$ must be an \emph{even} number since an odd number would require a \symop[h]{\sigma}.
    \item[\PG{T}{d}] The tetrahedral point group contains three mutually perpendicular \symop[2]C axes, four \symop[3]C axes and a centre of symmetry.
    \item[\PG{O}{h}] The octahedral point group has three mutually perpendicular \symop[4]C axes, four \symop[3]C axes and a centre of symmetry.
\end{itemize}
\endgroup

Determining the point group of a molecule is the first step in a treatment of the molecular orbitals or spectra of a compound.
It is therefore important that this be done somewhat systematically.
The flow chart in \fref{fig:flowchart} is offered as an aid, and a few examples should clarify the process.
We will first determine the point groups for the following \ce{Pt(II)} ions:

\begin{figure}[!htbp]
    \centering
    \schemestart[][south]
    \chemname{\chemleft[\chemfig{%
        Pt(-[1]\clclr{5}?(-[::45,,1,1,dotted]C_2)(-[-2,,1,1,dotted]))(-[::0,,1,1,dotted])(-[4,1.1,1,1,dotted])(-[-3]\clclr{5})(-[3]\clclr{5})-[::-45]\clclr{5}(-[::135,,1,1,dotted])(-[-2,,1,1,dotted])-[::90]Pt?(-[::-90]\clclr{5})(-[::0]\clclr{5})(-[0,1.1,1,1,dotted] C_2)(-[4,,1,1,dotted])}\chemright]%
    }{\bfseries A}
    \arrow{0}
    \chemname{\chemleft[\chemfig{%
        Pt(-[1]\clclr{5}?(-[::45,,1,1,dotted]C_2)(-[-2,,1,1,dotted]))(-[-3]\clr{Br}{1})(-[3]\clclr{5})-[::-45]\clclr{5}(-[::135,,1,1,dotted])(-[-2,,1,1,dotted])-[::90]Pt?(-[::-90]\clr{Br}{1})(-[::0]\clclr{5})}\chemright]%
    }{\bfseries B}
    \arrow{0}
    \chemname{\chemleft[\chemfig{%
        Pt(-[1]\clclr{5}?)(-[-3]\clr{Br}{1})(-[3]\clclr{5})-[::-45]\clclr{5}-[::90]Pt?(-[::-90]\clclr{5})(-[::0]\clr{Br}{1})}\chemright]%
    }{\bfseries C}
    \schemestop
\end{figure}

\textbf{A} contains three \symop[2]C axes, such that $\left[C_k?\right]$ is true for $k=2$.
It contains a plane of symmetry, such that $\left[\sigma?\right]$ is also true.
The three \symop[2]C axes are perpendicular, i.e. there is a \symop[2]C axis and two perpendicular $\symop[2]C'$'s which implies that $\left[\perp C_2?\right]$ is true.
Finally, there is a plane of symmetry perpendicular to the \symop[2]C, thus $\left[\perp\sigma?\right]$ is also true.
Combining these four observations, we arrive at the point group \PG{D}{2h}.
\textbf{B} contains only one \symop[2]C axis, no $\perp\symop[2]C$'s, no \symop[h]{\sigma}, but it does have two $\symop[v]{\sigma}$'s---making it a \PG{C}{2v} ion.
\textbf{C} contains a single \symop[2]C axis, and a horizontal plane (i.e. the plain of the ion) therefore has \PG{C}{2h} symmetry.

% Flowchart config
\usetikzlibrary{shapes, arrows, positioning, calc}
% We need a long box for header, circle for final point group, diamond for question
\tikzstyle{header} = [
    rectangle,
    text width=0.35\textwidth,
    minimum height=1em,
    text centered,
    draw=cmap2,
    fill=white!85!cmap2
]
% diamond but stretched
\tikzstyle{spg} = [
    ellipse,
    text width=0.2\textwidth,
    text centered,
    minimum height=1em,
    x radius = 3em,
    y radius = 2.5em,
    draw=cmap1, fill=white!85!cmap1,
    rounded corners,
]
\tikzstyle{pgrp} = [
    circle,
    text centered,
    draw=cmap1,
    fill=white!85!cmap1
]
\tikzstyle{question} = [
    diamond,
    minimum width=3.3em,
    minimum height=3.3em,
    text centered,
    draw=cmap4, 
    fill=white!85!cmap4
]
\tikzstyle{arrow} = [->, >=stealth, thick]
\def\childxdist{5.4em}
\def\childydist{1.8em}

\begin{figure}[!hbtp]
    \centering
    \begin{tikzpicture}
        \node (header) [header] {Special Point Groups?\\ \small (Linear, Tetrahedral,\\ Octahedral, Icosahedral)};
        \node (special) [spg, right of = header, xshift = 3 * \childxdist] {\PG{C}{\infty v}, \PG{D}{\infty v}, \PG{T}{}, \PG{T}{d},\\ \PG{T}{h}, \PG{O}{}, \PG{O}{h}, \PG{I}{h}};
        \draw[arrow] (header) -- node[midway,above] {Yes} (special);

        \node (ck) [question, below of = header, yshift = -1.5 * \childydist] {\PG{C}{k}?};
        \draw[arrow] (header) -- node[midway,left] {No} (ck);
            \node (sigmaN) [question, right of = ck, xshift = \childxdist] {\PG{\sigma}{?}}; \draw[arrow] (ck) -- node[midway,above] {No} (sigmaN);
                \node (cs) [pgrp, right of = sigmaN, xshift = \childxdist, yshift = \childydist] {\PG{C}{s}}; \draw[arrow] (sigmaN) -- node[midway,above] {Yes} (cs);
                \node (inv) [question, right of = sigmaN, xshift = \childxdist, yshift = -\childydist] {\PG{i}{}?}; \draw[arrow] (sigmaN) -- node[midway,below] {No} (inv);
                    \node (ci) [pgrp, right of = inv, xshift = \childxdist, yshift = \childydist] {\PG{C}{i}}; \draw[arrow] (inv) -- node[midway,above] {Yes} (ci);
                    \node (cone) [pgrp, right of = inv, xshift = \childxdist, yshift = -\childydist] {\PG{C}{1}}; \draw[arrow] (inv) -- node[midway,below] {No} (cone);

            \node (sigmaY) [question, below of = ck, yshift = -1.5 * \childydist] {\PG{\sigma}{}?}; \draw[arrow] (ck) -- node[midway,left] {Yes} (sigmaY);
                \node (stwok) [question, right of = sigmaY, xshift = \childxdist] {\PG{S}{2k}?}; \draw[arrow] (sigmaY) -- node[midway,above]{Yes} (stwok);
                    \node (stwokY) [pgrp, right of = stwok, xshift = \childxdist] {\PG{S}{2k}}; \draw[arrow] (stwok) -- node[midway,above] {Yes} (stwokY);
            
            \node (Tc2) [question, below of = sigmaY, yshift = -\childydist, xshift = 0.8 * \childxdist] {$\perp \PG{C}{2}?$};
            \draw[arrow] (sigmaY) -- node[midway,left] {Yes} (Tc2);
            \draw[arrow] (stwok) -- node[midway,right] {No} (Tc2);

            \node (sigmahY) [question, below of = Tc2, xshift = -0.8 * \childxdist, yshift = -\childydist] {\PG{\sigma}{h}?}; \draw[arrow] (Tc2) -- node[midway,left] {Yes} (sigmahY);
                \node (dkh) [pgrp, left of = sigmahY, xshift = -0.666 * \childxdist] {\PG{D}{kh}}; \draw[arrow] (sigmahY) -- node[midway,above] {Yes} (dkh);
                \node (sigmavY) [question, below of = sigmahY, yshift = -1.5 * \childydist] {\PG{\sigma}{v}?}; \draw[arrow] (sigmahY) -- node[midway, right] {No} (sigmavY);
                    \node (dkd) [pgrp, left of = sigmavY, xshift = -0.666 * \childxdist] {\PG{D}{kd}}; \draw[arrow] (sigmavY) -- node[midway,above] {Yes} (dkd);
                    \node (dk) [pgrp, below of = sigmavY, yshift = -1.25 * \childydist] {\PG{D}{k}}; \draw[arrow] (sigmavY) -- node[midway,right] {No} (dk);
            \node (sigmahN) [question, below of = Tc2, xshift = 0.8 * \childxdist, yshift = -\childydist] {\PG{\sigma}{h}?}; \draw[arrow] (Tc2) -- node[midway,right] {No} (sigmahN);
                \node (ckh) [pgrp, right of = sigmahN, xshift = 0.666 * \childxdist] {\PG{C}{kh}}; \draw[arrow] (sigmahN) -- node[midway,above] {Yes} (ckh);
                \node (sigmavN) [question, below of = sigmahN, yshift = -1.5 * \childydist] {\PG{\sigma}{v}?}; \draw[arrow] (sigmahN) -- node[midway, right] {No} (sigmavN);
                    \node (ckv) [pgrp, right of = sigmavN, xshift = 0.666 * \childxdist] {\PG{C}{kv}}; \draw[arrow] (sigmavN) -- node[midway,above] {Yes} (ckv);
                    \node (ck) [pgrp, below of = sigmavN, yshift = -1.25 * \childydist] {\PG{C}{k}}; \draw[arrow] (sigmavN) -- node[midway,right] {No} (ck);   
    \end{tikzpicture}
    \caption{Flowchart for the determination of molecular point groups}\label{fig:flowchart}
\end{figure}
\pagebreak

Next, we determine the point group to which the \keyword{staggered} and \keyword{eclipsed} firms of octa\-chloro\-di\-rhenate belong.

\begin{figure}[!htbp]
    \centering
    \schemestart[][center]
    \chemname{\chemfig[cram width = .35em, cram dash width = 0.8pt, cram dash sep = 3pt]{%
        Re(<:[1,1.5,,,white!75!black]\clf{Cl}{cmap5!50!white})(<[3,1.5]\clf{Cl}{cmap5!50!white})(<[5,1.5]\clf{Cl}{cmap5!50!white})(<:[7,1.5,,,white!75!black]\clf{Cl}{cmap5!50!white})-[::-15,2,,,white!75!black]%
        Re(-[::105,1.5,,,white!40!black]\clclf3)(-[::-75,1.5,,,white!40!black]\clclf3)(<[::-155,1.5]\clclf3)(<:[::20,1.5]\clclf3)}}{\bfseries Staggered}
    \arrow{0}
    \chemname{\chemfig[cram width = .35em, cram dash width = 0.8pt, cram dash sep = 3pt]{%
    Re(<:[1,1.5]\clf{Cl}{cmap5!50!white})(<[3,1.5]\clf{Cl}{cmap5!50!white})(<[5,1.5]\clf{Cl}{cmap5!50!white})(<:[7,1.5,,,white!75!black]\clf{Cl}{cmap5!50!white})-[::-15,2,,,white!75!black]%
    Re(<:[1,1.5]\clclf3)(<[3,1.5]\clclf3)(<[5,1.5]\clclf3)(<:[7,1.5,,,white!75!black]\clclf3)}}{\bfseries Eclipsed}
    \schemestop
\end{figure}


Both forms of \recl\ contain a \symop[4]C axis (\ce{Re-Re} bond) hence the answer to $\left[\PG{C}{k}?\right]$ is true for $k = 4$.
Both also contain planes of symmetry, thus $\left[\sigma ?\right]$ is also true.
The four $\perp \symop[2]C$'s passing through the centre of the \ce{Re-Re} bond can be observed as well.
This is more apparent for the eclipsed form---2 parallel to the \ce{Re-Cl} bonds and 2 parallel to their bisectors---unlike the staggered form whose $\perp \symop[2]C$'s bisecting the \ce{Cl-Re-Re-Cl} dihedral angles are shown as below:

\begin{center}
    \schemestart[][]
        \chemfig[cram width = .35em, cram dash width = 0.8pt, cram dash sep = 3pt]{%
        Re%
        % positive (frontal) bonds
        (<[0,2]\clclf3)(<[2,2]\clclf3)(<[4,2]\clclf3)(<[6,2]\clclf3)%
        % negative bonds
        (<:[1,2,,,white!40!black]\clf{Cl}{cmap5!50!white})(<:[3,2,,,white!40!black]\clf{Cl}{cmap5!50!white})(<:[5,2,,,white!40!black]\clf{Cl}{cmap5!50!white})(<:[7,2,,,white!40!black]\clf{Cl}{cmap5!50!white})%
        % C2 axes
        (-[0.5,4,,,dotted, cmap1!50!black]\clr{\symop[2]C}{1!50!black})%
        (-[4.5,4,,,dotted, cmap1!50!black])%
        (-[-0.5,4,,,dotted, cmap2!50!black]\clr{\symop[2]C}{2!50!black})%
        (-[3.5,4,,,dotted, cmap2!50!black])%
        (-[1.5,4,,,dotted, cmap4!50!black]\clr{\symop[2]C}{4!50!black})%
        (-[-2.5,4,,,dotted, cmap4!50!black])%
        (-[-1.5,4,,,dotted, cmap5!50!black]\clr{\symop[2]C}{5!50!black})%
        (-[2.5,4,,,dotted, cmap5!50!black])%
        }
    \schemestop
\end{center}

Both forms also contain vertical planes, but the eclipsed form also has a horizontal plane which is not present in the staggered form.
The point groups are, therefore, \PG{D}{4h} for the eclipsed form and \PG{D}{4d} for the staggered form.
It is crucial to be proficient with this process, and only \emph{practice makes perfect}.
\pagebreak

\begin{problem}
Determine the point group to which each of the following belongs:
\tcblower
\vbox{\tabskip=1em\offinterlineskip
    \ialign{\tabskip=1.2em
        \hfil\schemestart[][center]#\schemestop\hfil&
        \hfil\schemestart[][center]#\schemestop\hfil&
        \hfil\schemestart[][center]#\schemestop\hfil&
        \hfil\schemestart[][center]#\schemestop\hfil&
        \hfil\schemestart[][center]#\schemestop\hfil&
        \hfil\schemestart[][center]#\schemestop\hfil\cr
        \chemfig{C(-[::180]H)~N}&
        \chemfig[cram width = .35em, cram dash width = 0.8pt, cram dash sep = 3pt]{Sn(-[3,1.3]Cl)(-[-3,1.3]Cl)(<[::-20,1.3]Br)(<:[::20,1.3]Br)}&
        \chemfig[cram width = .35em, cram dash width = 0.8pt, cram dash sep = 3pt]{Sn(-[3,1.3]Cl)(-[-3,1.3]Cl)(<[::-20,1.3]Br)(<:[::20,1.3]Cl)}&
        \chemfig[cram width = .3em, cram dash width = 0.8pt, cram dash sep = 3pt]{?<[1.333,1.1]-[::-120,1.1,,,line width=.3em]>[::120,1.1]-[-2,0.8]-[::-150]-[::120]-[::-120]?}&
        \chemfig{Xe(-F)(-[::180]F)}&
        \chemfig[cram width = .35em, cram dash width = 0.8pt, cram dash sep = 3pt]{W(=[2]Se)(=[-2]Se)(<[::-20,1.3]P)(<:[::20,1.3]P)(<[::160,1.3]P)(<:[::-160,1.3]P)}\cr\noalign{\bigskip}
        \chemfig[cram width = .35em, cram dash width = 0.8pt, cram dash sep = 3pt]{Pt(<[::-20,1.3]Cl)(<[::-160,1.3]Br)(<:[::20,1.3]Cl)(<:[::160,1.3]Br)(-[2]Cl)(-[-2]Br)}&
        \chemfig[cram width = .35em, cram dash width = 0.8pt, cram dash sep = 3pt]{Pt(<[::-20,1.3]Cl)(<[::-160,1.3]Br)(<:[::20,1.3]Br)(<:[::160,1.3]Br)(-[2]Cl)(-[-2]Cl)}&
        \chemfig[cram width = .35em, cram dash width = 0.8pt, cram dash sep = 3pt]{Pt(<[::-20,1.3]Cl)(<[::-160,1.3]Br)(<:[::20,1.3]Cl)(<:[::160,1.3]Br)(-[2]Cl)(-[-2]Cl)}&
        \chemfig[cram width = .35em, cram dash width = 0.8pt, cram dash sep = 3pt]{Re(<[::-20,1.3]CN)(<[::-160,1.3,,2]NC)(<:[::20,1.3]CN)(<:[::160,1.3,,2]NC)(=[2]O)(=[-2]O)}&
        \chemfig[cram width = .35em, cram dash width = 0.8pt, cram dash sep = 3pt]{V(<[::-20]@{od}O)(<[::-160]@{oa}O)(<:[::20]@{oc}O)(<:[::160]@{ob}O)(=[2]O)(-[-2]py)}%
        \chemmove{\draw[-](oa)..controls ++(170:1em) and ++(-170:1em)..(ob);}%
        \chemmove{\draw[-](oc)..controls ++(-10:1em) and ++(10:1em)..(od);}&
        \chemfig[cram width = .35em, cram dash width = 0.8pt, cram dash sep = 3pt]{V(<[::-20]@{ob}O)(<[::-160]@{oc}O)(<:[::20]@{oa}O)(<:[::160]py)(=[2]O)(-[-2]@{od}O)}%
        \chemmove{\draw[-](oa)..controls ++(-10:1em) and ++(10:1em)..(ob);}%
        \chemmove{\draw[-](od)..controls ++(-140:1.25em) and ++(-110:1.25em)..(oc);}
        \cr\noalign{\bigskip}
        \chemfig[cram width = .35em, cram dash width = 0.8pt, cram dash sep = 3pt]{%
            Ru(<[::-20,1.3]CN)(<:[::20,1.3]CN)%
            (-[2]N?[na])(<:[::160,1.3]N-[2,0.7]-[::-65,0.7]?[na])%
            (-[-2]N?[nb])(<[::-160,1.3]N-[-2,0.65]-[::65,0.7]?[nb])%
        }&
        \chemfig[cram width = .35em, cram dash width = 0.8pt, cram dash sep = 3pt]{Mo(-[::90]N)(<:[::-15,1.3]Cl)(<[::-55,1.3]O)(<[::-120,1.3]Cl)(<:[::-165,1.3]O)}&
        \chemfig[cram width = .35em, cram dash width = 0.8pt, cram dash sep = 3pt]{Cu(-[3,1.3]I)(-[-3,1.3]I)(<[::-20,1.3]Br)(<:[::20,1.3]Cl)}&
        \chemfig[cram width = .35em, cram dash width = 0.8pt, cram dash sep = 3pt]{%
            (-[::0,0.8](-[::90]Cl)(-[::-90]Cl)(<[::-20]F)(<:[::160]F))%
            (-[::180,0.8](<[::120]F)(<[::-120]Cl)(<:[::60]Cl)(<:[::-60]F))%
        }&
        \chemfig[cram width = .35em, cram dash width = 0.8pt, cram dash sep = 3pt]{%
            (-[::90]F)(-[0,1.2]F)(-[::-90]F)(<:[::160,1.2]F)(<[::-160,1.2]Cl)
        }&
        \chemfig{Sb(=[0,1.1]O)-[::-150,1.1]Cl}\cr
    }
}
\end{problem}
\section{Matrix Representations of Groups}

How may one utilise matrices to represent the symmetry operations?
Seemingly there are infinite ways.
The choice of representation is determined by its \keyword{basis} i.e. by th elabels or functions attached to objects.
The number of basis functions, or \keyword{label}\textbf{s}, is called the \keyword{dimension} of the representation.

A convenient basis to use when dealing with the motions of molecules is the set of Cartesian displacement vectors.
Each atom has three degrees of motional freedom, meaning a molecule with $N$ number of atoms will have a basis of dimension $3N$.
For example, each operation on a water molecule---having a 9-dimensional basis---can be represented by a $9\times9$ matrix.
Below shows these 9-basis vectors along with the results of the $\symop[2]C (z)$ rotation:

\begin{center}
    \newcommand{\bigone}{\chemskipalign\tikz\node[draw, circle, minimum size=1.75em, inner sep=0pt, fill=cmap1] (0,0) {\textcolor{white}{O}};}
    \newcommand{\bigother}[1]{\chemskipalign\tikz\node[atom, fill=cmap#1] (0,0) {\textcolor{white}{H}};}
    \schemestart[][center]
    \chemfig[cram width = .3em, cram dash width = 0.4pt, cram dash sep = 2pt]{%
        \bigone(-[2,1.35]z_1)(-[0,1.35]y_1)(<[-0.5,1.35]x_1)%
        (-[::-36,2.5]\bigother{2}(-[2,1.35]z_3)(-[0,1.35]y_3)(<[-0.5,1.35]x_3))%
        (-[::-144,2.5]\bigother{4}(-[2,1.35]z_2)(-[0,1.35]y_2)(<[-0.5,1.35]x_2))%
    }
    \arrow{->[\symop[2]C]}
    \chemfig[cram width = .3em, cram dash width = 0.4pt, cram dash sep = 2pt]{%
        \bigone(-[2,1.35]z_1)(-[4,1.35]y_1)(<:[3.5,1.35]x_1)%
        (-[::-36,2.5]\bigother{4}(-[2,1.35]z_2)(-[4,1.35]y_2)(<:[3.5,1.35]x_2))%
        (-[::-144,2.5]\bigother{2}(-[2,1.35]z_3)(-[4,1.35]y_3)(<:[3.5,1.35]x_3))%
    }
    \schemestop
    \let\bigone\undefined
    \let\bigother\undefined
\end{center}

The result of this operation is: \(\left(x_i \to -x_j\right)\), \(\left(y_i \to -y_j\right)\) and \(\left(z_i \to +z_j\right)\) where $i = j$ for the \ce{O} atom coordinates since the \ce{O} lies on the \symop[2]C axis and therefore does not change its position.
However, $i \neq j$ for the hydrogen atoms because they do not lie on the \symop[2]C axis---hence rotated into one another e.g. \(\left(x_2 \to -x_3\right)\).
Such transformation can be represented in matrix notation where each atom will have a $3\times3$ matrix:

\begin{equation*}
    \begin{bmatrix}
        x_i \\ y_i \\ z_i
    \end{bmatrix} = \begin{bmatrix}
        -1 & 0 & 0 \\
        0 & -1 & 0 \\
        0 & 0 & +1
    \end{bmatrix} \begin{bmatrix}
        x_j \\ y_j \\ z_j
    \end{bmatrix}
\end{equation*}

\noindent which must be placed into the $9\times9$ matrix representation of the \symop[2]C operation.
The \ce{O} atom is not affected by the rotation $(i = j = 1)$ thus its $3\times3$ matrix remains in its original position $(1, 1)$ on the diagonal.
Meanwhile, the \ce{H} atoms are exchanged by the  so their matrices are rotated off of the diagonal to the $(2, 3)$ and $(3, 2)$ positions.
To represent the \symop[2]C rotation in matrix notation:

\begin{equation*}
C_2 \begin{bmatrix} x_1\\ y_1\\ z_1\\ x_2\\ y_2\\ z_2\\ x_3\\ y_3\\ z_3 \end{bmatrix} = \begin{bNiceMatrix}[r,left-margin=0.5em, right-margin=0.5em]
    -1 & 0 & 0 & 0 & 0 & 0 & 0 & 0 & 0 \\
    0 & -1 & 0 & 0 & 0 & 0 & 0 & 0 & 0 \\
    0 & 0 & +1 & 0 & 0 & 0 & 0 & 0 & 0 \\
    0 & 0 & 0 & 0 & 0 & 0 & -1 & 0 & 0 \\
    0 & 0 & 0 & 0 & 0 & 0 & 0 & 1 & 0 \\
    0 & 0 & 0 & 0 & 0 & 0 & 0 & 0 & +1 \\
    0 & 0 & 0 & -1 & 0 & 0 & 0 & 0 & 0 \\
    0 & 0 & 0 & 0 & -1 & 0 & 0 & 0 & 0 \\
    0 & 0 & 0 & 0 & 0 & +1 & 0 & 0 & 0
    \CodeAfter
    \SubMatrix[{1-1}{3-3}]
    \SubMatrix[{4-7}{6-9}]
    \SubMatrix[{7-4}{9-6}]
\end{bNiceMatrix} \begin{bmatrix} x_1\\ y_1\\ z_1\\ x_2\\ y_2\\ z_2\\ x_3\\ y_3\\ z_3 \end{bmatrix}
\end{equation*}

In the case of reflection through the plane of the molecule i.e. \(\left(\symop[v]{\sigma} = YZ\right)\), only the $x$-coordinate is changed and no atoms are transformed, thus the matrix representation is:

\begin{equation*}
    \sigma \begin{bmatrix} x_1\\ y_1\\ z_1\\ x_2\\ y_2\\ z_2\\ x_3\\ y_3\\ z_3 \end{bmatrix} = \begin{bNiceMatrix}[r,left-margin=0.5em, right-margin=0.5em]
    -1 & 0 & 0 & 0 & 0 & 0 & 0 & 0 & 0 \\
    0 & +1 & 0 & 0 & 0 & 0 & 0 & 0 & 0 \\
    0 & 0 & +1 & 0 & 0 & 0 & 0 & 0 & 0 \\
    0 & 0 & 0 & -1 & 0 & 0 & 0 & 0 & 0 \\
    0 & 0 & 0 & 0 & +1 & 0 & 0 & 0 & 0 \\
    0 & 0 & 0 & 0 & 0 & +1 & 0 & 0 & 0 \\
    0 & 0 & 0 & 0 & 0 & 0 & -1 & 0 & 0 \\
    0 & 0 & 0 & 0 & 0 & 0 & 0 & +1 & 0 \\
    0 & 0 & 0 & 0 & 0 & 0 & 0 & 0 & +1
    \CodeAfter
    \SubMatrix[{1-1}{3-3}]
    \SubMatrix[{4-4}{6-6}]
    \SubMatrix[{7-7}{9-9}]
\end{bNiceMatrix} \begin{bmatrix} x_1\\ y_1\\ z_1\\ x_2\\ y_2\\ z_2\\ x_3\\ y_3\\ z_3 \end{bmatrix}
\end{equation*}

Thus, $9\times9$ matrices, as shown above, can serve as a method to represent operations for the water molecule in this basis.
Thankfully, we need only specify the \keyword{trace} of this matrix i.e. the \textit{sum} of the diagonal elements.
The resulting number is called the \keyword{character}, and the character of an operation $\mathbf R$ is given by the symbol \charop{\mathbf R}.
For example, $\charop{\symop[2]C} = -1 $ and $\charop{\symop[v]{\sigma}} = 3$.
An observant student may notice two important points about the character:
\begin{enumerate}
    \item Only the atoms that remain in the same position can contribute to the trace, otherwise their matrices will be rotated off the diagonal;
    \item Each operation contributes the same amount to the trace for each atom because all atoms have the same matrix.
\end{enumerate}
Indeed so. For a reflection through the plane bisecting the \ce{H-O-H} bond angle, $\charop{\symop[v]{\sigma'}} = +1$ since the \ce{O}---and only \ce{O}---is unshifted and a plane contributed $+1$ for each unshifted atom.
Additionally, the character for the identity element will always be the dimension of the basis since all labels are unchanged. For water molecule, $\charop{\symop{E}} = 9$.

Suppose we were to summate this information and put it into a tabular format. The representation $\left(\Gamma\right)$ for water molecule in this basis is:
\begin{center}
$\begin{NiceArray}{*{5}{c}}[hvlines]
    {} & \symop{E} & \symop[2]C & \symop[v]{\sigma}' (YZ) & \symop[v]{\sigma} (XZ)\\
    \Gamma & 9 & -1 & 3 & 1
\end{NiceArray}$
\end{center}

The $s$-orbitals can also serve as a basis. In such a basis, each atom has $1$ and not $3$ labels which, in turn, makes each operation a $3\times3$ matrix cf. $9\times9$ in the previous basis i.e. that of \porb*.

\begin{center}
    \newcommand{\bigone}{\chemskipalign\tikz\node[draw, circle, minimum size=1.75em, inner sep=0pt, fill=cmap1] (0,0) {\textcolor{white}{I}};}
    \newcommand{\bigother}[2][3]{\chemskipalign\tikz\node[atom, fill=cmap#1] (0,0) {\textcolor{white}{#2}};}
    \schemestart[][]
    \chemfig{\bigone(-[::-30,2]\bigother{III})(-[::-150,2]\bigother[4]{II})}
    \schemestop
    \let\bigone\undefined
    \let\bigother\undefined
\end{center}

Further, there can be no sign change for an $s$-orbital.
That is:

\begin{equation*}
    \begin{bNiceArray}{c}[last-row,delimiters/color=white]
        \\
        \\
        \\
        \Gamma
    \end{bNiceArray}\quad%
    E = \begin{bNiceArray}{ccc}[last-row]
        1 & 0 & 0 \\
        0 & 1 & 0 \\
        0 & 0 & 1 \\
        &   & 3
    \end{bNiceArray}\quad%
    \symop[2]C = \begin{bNiceArray}{ccc}[last-row]
        1 & 0 & 0 \\
        0 & 0 & 1 \\
        0 & 1 & 0 \\
        &   & 1
    \end{bNiceArray}\quad%
    \symop[v]{\sigma} = \begin{bNiceArray}{ccc}[last-row]
        1 & 0 & 0 \\
        0 & 1 & 0 \\
        0 & 0 & 1 \\
        &   & 3
    \end{bNiceArray}\quad%
    \symop[v]{\sigma'} = \begin{bNiceArray}{ccc}[last-row]
        1 & 0 & 0 \\
        0 & 0 & 1 \\
        0 & 1 & 0 \\
        &   & 1
    \end{bNiceArray}
\end{equation*}

If one basis $\left(f'\right)$ is a linear combination of another basis $\left(f\mathop{}\right)$, or $f' = Cf$, then the representation in one basis should be \emph{similar} to each other.
It can be shown that the matrix representations of operator $\mathbf R$ in these two basis sets $\left(D\mathop{}(\mathbf R)\textrm{ and }D'(\mathbf R)\right)$ are related by a similarity transformation $D'(R) = C^{-1}D\mathop{}(R)\mathop{}C$ and $D\mathop{}(R) = CD'(R)\mathop{}C^{-1}$.
That is, the matrices $D\mathop{}(R)$ and $D'(R)$ are conjugate.
For example, the linear combination of \sorb{}s: $X = \mathrm{\color{cmap1}I} + \mathrm{\color{cmap4}II}$, $Y = \mathrm{\color{cmap1}I} + \mathrm{\color{cmap3}III}$ and $Z = \mathrm{\color{cmap4}II} + \mathrm{\color{cmap3}III}$  can be expressed in matrix form as:

\begin{equation*}
    \begin{bNiceMatrix}[first-row]
        f'\\
        X\\
        Y\\
        Z
    \end{bNiceMatrix}
    \begin{bNiceMatrix}[first-row,delimiters/color=white]
        =\\
        {}\\
        =\\
        {}
    \end{bNiceMatrix}
    \begin{bNiceMatrix}[first-row]
        &C&\\
        1&1&0\\
        1&0&1\\
        0&1&1
    \end{bNiceMatrix}\,
    \begin{bNiceMatrix}[first-row]
        f\\
        \mathrm{I}\\
        \mathrm{II}\\
        \mathrm{III}
    \end{bNiceMatrix}%
    \quad\text{and}\quad%
    C^{-1} = \frac{1}{2}\cdot\begin{bNiceMatrix}
        +1&+1&-1\\
        +1&-1&+1\\
        -1&+1&+1
    \end{bNiceMatrix}
\end{equation*}

\noindent The matrix representation of \(\symop[2]C \left((D'\left(C_2\right)\right)\) in the new basis is then give by \(D'\left(C_2\right) = C^{-1}\mathop{}D\left(C_2\right)\mathop{}C\), or %
\(
    D'(C_2) = \frac{1}{2} \begin{bNiceMatrix}
        +1&+1&-1\\
        +1&-1&+1\\
        -1&+1&+1
    \end{bNiceMatrix} \begin{bNiceMatrix}
        1&0&0\\
        0&0&1\\
        0&1&0
    \end{bNiceMatrix} \begin{bNiceMatrix}
        1&1&0\\
        1&0&1\\
        0&1&1
    \end{bNiceMatrix} = \begin{bNiceMatrix}
        0&1&0\\
        1&0&0\\
        0&0&1
    \end{bNiceMatrix}
\).
Similarly, \(D'(\symop[v]{\sigma'}) = \begin{bNiceMatrix}
    0&1&0\\
    1&0&0\\
    0&0&1
\end{bNiceMatrix}\) and \(D'(\symop[v]{\sigma}) = D'(\symop{E}) = \begin{bNiceMatrix}
    1&0&0\\
    0&1&0\\
    0&0&1
\end{bNiceMatrix}\).

Note that the representation for \symop[2]C and \symop{\sigma'} have changed, but in all cases the \textbf{character is invariant with the similarity transformation}.
That is, all members of a class of operations can be treated altogether since they are related by a similar transformation and therefore also must have the same characters.
The two $3\times3$ bases used to to this point can be viewed as consisting of a $1\times1$ matrix---where one basis vector is not rotated into any of the others by any operation of the group, $I$ \& $Z$\/--- and a $2\times2$ sub-matrix---where two basis vectors are rotated into each other by at least one operation of the group.
Thus, the $3\times3$ matrix representation ohas been reduced to a $1\times1$ matrix and a $2\times2$ matrix.
Indeed, the $2\times2$ matrix can be reduced into two $1\times1$ matrices as well.
In such processes, a large \keyword{reducible representation} is \emph{decomposed} into smaller---typically $1\times1$ but also $2\times2$ or $3\times3$---\keyword{irreducible representation}\textbf{s}.

Consider the \keyword{symmetry adapted linear combination}\textbf{s} (SALC’s) represented by $A = \mathrm{\color{cmap1}I}$, $B = \mathrm{\color{cmap4}II} +  \mathrm{\color{cmap3}III}$ and $C = \mathrm{\color{cmap4}II} - \mathrm{\color{cmap3}III}$.

\begin{center}
    \newcommand{\bigone}{\chemskipalign\tikz\node[draw, circle, minimum size=1.75em, inner sep=0pt, fill=cmap1] (0,0) {};}
    \newcommand{\bigempty}{\chemskipalign\tikz\node[draw, circle, minimum size=1.35em, inner sep=0pt, fill=white, color=white] (0,0) {};}
    \newcommand{\bigother}[1]{\chemskipalign\tikz\node[atom, fill=cmap#1] (0,0) {};}
    \schemestart[][]
        \chemname{%
            \chemfig{\bigone(-[::-30,1.5])(-[::-150,1.5])}
        }{\bfseries A}
        \arrow{0}
        \chemname{\chemfig{\bigempty(-[::-30,1.5]\bigother{1})(-[::-150,1.5]\bigother{1})}}{\bfseries B}
        \arrow{0}
        \chemname{\chemfig{\bigempty(-[::-30,1.5]\bigother{2})(-[::-150,1.5]\bigother{1})}}{\bfseries C}
    \schemestop
\end{center}

\noindent In this basis, \symop[2]C rotation and a reflection through the plane orthogonal to the molecular plane do not change $A$ nor $B$, and only change the sign of $C$ while reflection through the molecular plane leaves all three unchanged.
That is, %
\(
    E = \sigma = \begin{bNiceMatrix}
        1&0&0\\
        0&1&0\\
        0&0&1
    \end{bNiceMatrix}
\)
while %
\(
    C_2 = \sigma' = \begin{bNiceMatrix}
        0&1&0\\
        1&0&0\\
        0&0&-1
    \end{bNiceMatrix}
\).

Note how no basis vector is changed into another by a symmetry operation, as in this basis is \emph{symmetry adapted}.
That is, our $3\times3$ representation now consists of three $1\times1$ matrices, and we have converted our reducible representation $\Gamma$ into three \emph{irreducible} representations: $\Gamma_1$; $\Gamma_2$; and $\Gamma_3$ such that $\Gamma = \Gamma_1 \oplus \Gamma_2 \oplus \Gamma_3$.

\begin{center}
$\begin{NiceArray}{l *{4}{r}}[hvlines]
    {} & E & C_2 & \sigma & \sigma' \\
    \Gamma_1 & 1 & 1 & 1 & 1 \\
    \Gamma_2 & 1 & 1 & 1 & 1 \\
    \Gamma_3 & 1 & -1 & 1 & -1\\
    \Gamma\  & 3 & 1 & 3 & 1
\end{NiceArray}$
\end{center}

The term “irreducible representation” is used very frequently, reasons for which we will find out soon.
Decomposing a reducible representation into irreducible representations is an important process.
Detailed steps for decomposing reducible representations will be described in later sections.
\section{Point group representations}

A point group representation is a basis set in which the irreducible representations are the basis vectors.
As $\mathbf i$, $\mathbf j$, and $\mathbf k$ form a complete, orthonormal basis for three-3-dimensional space, so too do the irreducible representations form the a complete orthonormal basis for an $m$-dimensional space, where $m$ is the number of irreducible representations and is equal to the number of classes in the group.
These considerations are summarised by the following rules:

\begin{enumerate}
    \item The number of basis vectors, or irreducible representations $m$ equals the number of classes;
    \item The sum of the squares of the dimensions of the $m$ irreducible representations equals the order \[\sum_{i=1}^{m} d_i^2 = h\]
    The character of the identity operation equals the dimension of the representation \charop{E} = $d_i$, which is referred to as the \keyword{degeneracy} of the \irrep. The degeneracy of most \irrep s is $1$ i.e. non-degenerate representations are $1\times1$ matrices. but sometimes can also be $2$ or $3$.
    \ce{C60}---also known as buckminster fullerene or “Buckyball”---belongs in the icosahedral point group.
    It has \irrep{}s with degeneracies of $4$ and $5$.
    No character in an \irrep\ can exceed the dimension of the representation.
    Thus, in non-degenerate representations, all characters must be $\pm1$.
    \item The member \irrep{}s are orthonormal i.e. the sum of the squares of the characters in any \irrep\ is equal to the order (row normalisation), while the sum of the product of the characters over all operations in two different \irrep{}s is $0$ (orthogonality) \[\sum_R \mathop{g}(R)\mathop{\chi_i}(R)\mathop{\chi_j}(R) = \mathop{h} \delta_{ij}\] where the sum is over all of the \emph{classes} of operations, $\mathop{g}(R)4$ is the number of operation, $R$ is the class, $\mathop{\chi_i}(R)$ and $\mathop{\chi_j}(R)$ are the characters of operation $R$ in the $i$-th and $j$-th \irrep{}s, and $h$ is the order of the group and $\delta_{ij}$ is the Kronecker delta ($0$ when $i\neq j$ and $1$ when $i=j$).
    \item The sum of the squares of the characters of any operation over all of the \irrep{}s times the number of operations in the class is equal to $h$, as in columns of the representation are also normalised such that: \[\sum^{m}_{i=1} \mathop{g}(R)\mathop{\chi^2_i}(R) = h\] where $m$ is the number of \irrep{}s.
    \item The sum of the products of the characters of any two unequal operations over all of the \irrep{}s is $0$---columns of the representation are also orthogonal such that: \[\sum_{i=1}^{m} \mathop{\chi_i}(R')\mathop{\chi_i}(R) = 0\]
    \item The \emph{first} representation is always the totally symmetric representation in which all characters are $+1$.
    \item Any reducible representation in the point group can be expressed as a linear combination of the \irrep{}s---the completeness of the set.
\end{enumerate}

\subsection*{Generating the \texorpdfstring{\PG{C}{2v}}{C2v} point group}

Let us now generate the \PG{C}{2v} point group: $\left\{C_{2v} \middle| E, C_2, \sigma, \sigma'\right\}$.
The order of the group $(h)$ and the number of classes $(m)$ are both $4$.
Each class has only one operation i. e. $\mathop{g}(R) = 1$ in all cases.
Using rule 2, we can write \(d_1^2 + d_2^2 + d_3^2 + d_4^2 = 4\) so that $d_1 = d_2 = d_3 = d_4 = 1$---there are no degenerate representations in \PG{C}{2v}.
The character of the identity operation is always the dimension of the representation, so $\mathop{\chi_i}(E) = d_i$.
That is, all of the characters of the identity operation are $1$.
Inferring from rule 6, we may write the character of $\Gamma_1$ as $+1$.
In sum, we have the following character table for \PG{C}{2v}:
\begin{equation*}
    \begin{NiceArray}{c *{4}{r}}[hvlines]
        \PG{C}{2v} & E & C_2 & \sigma & \sigma' \\
        \Gamma_1 & 1 & 1 & 1 & 1 \\
        \Gamma_2 & 1 & a & b & c \\
        \Gamma_3 & 1 & d & e & f \\
        \Gamma_4 & 1 & g & h & i
    \end{NiceArray}
\end{equation*}
\noindent Since there are no degenerate representations, characters $\mathbf a$ through $\mathbf i$ are all $\pm1$.
To retain the orthogonality of rows and columns, only \emph{one} of the remaining characters in each row and in each column may be $+1$ while the other two in each column and row must each be $-1$.
That is, there are only three remaining $+1$'s and no two can be in the same column or row.
If we make $\mathbf a$, $\mathbf e$ and $\mathbf i$ as $+1$, then the rest must be $-1$.

With this, the \keyword{character table} for \PG{C}{2v} is then determined to be:
\begin{equation*}
    \begin{NiceArray}{c *{4}{r}}[hvlines]
        \PG{C}{2v} & E & C_2 & \sigma & \sigma' \\
        \Gamma_1 & 1 & 1 & 1 & 1 \\
        \Gamma_2 & 1 & 1 & -1 & -1 \\
        \Gamma_3 & 1 & -1 & 1 & -1 \\
        \Gamma_4 & 1 & -1 & -1 & 1
    \end{NiceArray}
\end{equation*}

\subsubsection*{Mulliken symbols}

In the \PG{C}{2v} point group, the \irrep{}s are designated as $\Gamma_1 = \mathup A_1$, $\Gamma_2 = \mathup B_1$, $\Gamma_3 = \mathup B_2$ and $\Gamma_4 = \mathup B_2$.
What we have done is to assign the \irrep{}s to the \keyword{Mulliken symbols} for the \irrep{}s.
Mulliken symbols for \irrep{}s are as follows:
\begin{itemize}
    \item[$\mathup A$] the \irrep\ is symmetric with respect to the rotation about the principle axis \(\charop{\mathop{C_n} (z)} = +1\)
    \item[$B$] the \irrep\ is antisymmetric with respect to the rotation about the principle axis \(\charop{\mathop{C_n} (z)} = -1\)
    \item[$\mathup E$] is a doubly degenerate representation \(d = 2 \implies \charop{E} = 2\)
    \item[$\mathup T$] is a triply degenerate representation \(d = 3 \implies \charop{E} = 3\)
    \item[$\mathup G$] implies degeneracy of 4
    \item[$\mathup H$] implies degeneracy of 5 
\end{itemize}
In many instances there are more than one A, B, E etc. \irrep{}s present in the point group.
This necessitates the use of subscripts and superscripts for further clarification:
\begin{itemize}
    \item[g/u] used in point groups with centres of symmetry $(i)$ to denote \textbf{g}erade\index[keywords]{gerade}\footnotemark[1] and \textbf{u}ngerade\index[keywords]{ungerade}\footnotemark[2] with respect to inversion.
    \item[${}'$\//\/${}''$] used to designate symmetric and antisymmetric with respect to inversion through a \symop[h]{\sigma} plane.
    \item Otherwise numerical subscripts are used.
\end{itemize}

\footnotetext[1]{German for ‘even’, implies symmetric}
\footnotetext[2]{German for ‘odd’, implies antisymmetric}

In the treatment of of molecular systems, one generates a reducible representation using an appropriate basis---such as
atomic orbitals or displacement vectors in Euclidean space---and then decomposes this reducible representation into its
component \irrep{}s to arrive at a description of the system which contains the information available from the
molecular symmetry.
There are certain symmetry properties which are very important and immutable so long as the point group remains the
same.
Since the symmetries of these various aspects are used frequently, they are also included in the character table.
We will now demonstrate the determination of these symmetries for the water molecule.

Since the \sorb\ on the \ce{O} atom lies on all of the symmetry elements and is spherically symmetric, it is unchanged
by a rotation about the \symop[2]C axis or reflection through the planes.
Thus the representation of the \sorb\ is:
\begin{equation*}
	\begin{NiceArray}{lcccc}[hvlines]
		\PG{C}{2v} & E & C_2 & \sigma & \sigma' \\
		\sorb & +1 & +1 & +1 & +1
	\end{NiceArray}
\end{equation*}
\noindent which is the $\mathup A_1$ \irrep.
The takeaway here is the \textbf{\sorb{}s on central elements will always transform as the totally symmetric representation but are not included in character tables.}

The three \porb{}s, translation along the $x$, $y$ and $z$ axes, and the three component of the \keyword*{electric dipole
operator} $\mu_x$, $\mu_y$ and $\mu_z$ all transform in the same way.
The \porb{x}s and \porb{y}s will change sign with a \symop[2]C operation an with reflection through the perpendicular
plane $YZ$ and $XZ$ respectively.
The \porb{z} is not affected by any of the symmetry operations.
Thus, $\Gamma_{\mathup p} = \mathup A_1 + \mathup B_1 + \mathup B_2$. The \porb{x} is said to\begin{itemize}
	\item form the basis for the $\mathup B_1$ representation,
	\item have $\mathup B_1$ symmetry or
	\item transform as $\mathup B_1$.
\end{itemize}
\noindent Translations along the $x$, $y$ and $z$-directions transform in the same way as $p_x$, $p_y$ and $p_z$.
To visualise this, simply translate the water molecule slightly along the $x$-axis without moving any of the symmetry
elements.
In this new position, \symop[2]C and $\sigma(yz)$ are destroyed and the characters are those of $p_x$ above.

Rotation of the water molecule slightly about the $z$-axis moves the \ce{H} atoms out of the plane.
In such orientation, the \symop[2]C axis is still preserved but both planes of symmetry are destroyed such that $\mathup R_z$ transforms to $\mathup A_2$.
Rotation about the $x$-axis preserves the $yz$ plane but destroys the \symop[2]C rotation and the $xz$ reflection while rotation about the $y$-axis preserves the $xz$ plane but destroys the \symop[2]C rotation and the $yz$ reflection.
\begin{equation*}
    \begin{NiceArray}{ccccc}[hvlines]
        \PG{C}{2v} & E & \symop[2]C (z) & \symop[v]{\sigma} (xz) & \symop[v]{\sigma} (yz) \\
        \mathup R_z & +1 & +1 & -1 & -1 \\
        \mathup R_x & +1 & -1 & -1 & +1 \\
        \mathup R_y & +1 & -1 & +1 & -1
    \end{NiceArray}
\end{equation*}
\noindent Thus, the rotations in the \PG{C}{2v} point group transform as $\mathup A_2 + \mathup B_1 + \mathup B_2$.
In most character tables, \PG{C}{2v} has the following form:
\begin{equation*}
    \begin{NiceArray}{ccccc || c || c}[hvlines]
        \PG{C}{2v} & E & \symop[2]C (z) & \symop[v]{\sigma} (xz) & \symop[v]{\sigma} (yz)\\
        \mathup A_1 & +1 & +1 & +1 & +1 & z & x^2, y^2, z^2\\
        \mathup A_2 & +1 & +1 & -1 & -1 & \mathup R_z & xy\\
        \mathup B_1 & +1 & -1 & +1 & -1 & x, \mathup R_y & xz\\
        \mathup B_2 & +1 & -1 & -1 & +1 & y, \mathup R_x & yz\\
    \end{NiceArray}
\end{equation*}
\noindent The final column yields the squares and binary products of the coordinates and represent the transformation properties of the \dorb{}s.

\subsection*{Generating the \texorpdfstring{\PG{C}{3v}}{C3v} point group}

As another example, we shall now generate the \PG{C}{3v} point group.
The operations of \PG{C}{3v} are \symop E, \symop[3]C, \symop[3]{C^2}, \symop[v]{\sigma}, \symop[v]{\sigma'} and \symop[v]{\sigma''}.
These can be simplified as \symop E, $2 \symop[3]C$ and $3 \symop[v]{\sigma}$ because \symop[3]C and the like are conjugate, as well as the three \symop[v]{\sigma}’s.
The \PG{C}{3v} group has an order of $6$ and contains three classes: $(h=6, m=3) \implies d_1^2 + d_2^2 + d_3^2 = 6 \implies d_1 = d_2 = 1$ and $d_3 = 2$.
Since the dimensions of the \irrep{}s are the \charop{E} and every group contains the totally symmetry \irrep:
\begin{equation*}
    \begin{NiceArray}{cccc}[hvlines]
        \PG{C}{3v} & \symop E & 2 \symop[3]C & 3 \symop[v]{\sigma} \\
        \Gamma_1 & 1 & 1 & 1 \\
        \Gamma_2 & 1 & j & k \\
        \Gamma_3 & 2 & m & n
    \end{NiceArray}
\end{equation*}
\noindent Orthogonality with $\Gamma_1$ requires that $\sum \mathop{g}(R) \charop{R} = 0$ for $\Gamma_2$ where $\left[1 \cdot 1 \cdot 1 + 2 \cdot l \cdot j + 3 \cdot l \cdot k = 0\right]$ and for $\Gamma_3$ where $\left[1\cdot 1 \cdot 2 + 2 \cdot l \cdot m + 3 \cdot l \cdot n = 0\right]$.
$\Gamma_2$ is non-degenerate i.e. $j$ and $k$ must each be $\pm 1$ hence the orthogonality condition implies that $j = +1$ and $k = -1$.
Normalisation of $\Gamma_3$ means $(1)(2^2) + 2(m^2) + 3(n^2) = 6$, which renders $m = -1$ and $n = 0$.
Alternatively, we can use the fact that $\mathop{g}(R) \sum \charop{R}^2 = h$ along any column, e.g. $2(1^2 + 1^2 + m^2) = 6\;\therefore m^2 = 1$ and $3\left(1^2 + (-1)^2 + n^2\right) = 6\;\therefore n^2 = 0$ and so on.
\begin{equation*}
    \begin{NiceArray}{c *{3}{r}}[hvlines]
        \PG{C}{3v} & \symop E & 2 \symop[3]C & 3 \symop[v]{\sigma} \\
        \Gamma_1 & 1 & 1 & 1 \\
        \Gamma_2 & 1 & 1 & -1 \\
        \Gamma_3 & 2 & -1 & 0
    \end{NiceArray}
\end{equation*}
We shall use the ammonia (\ce{NH3}) molecule as an example.
Coordinate system we will use in the following discussion is shown below.
\begin{figure}[!htbp]
    \centering
    \schemestart[][north]
        \chemfig[cram width = .3em, cram dash width = 0.75pt, cram dash sep = 2pt]{%
            N%
            (<:[::-170,1.25]\clr{H}{1})(<[::-145,1.25]\clr{H}{2})(-[::-30,1.25]\clr{H}{3})%
            (-[::0,1.75,,,dotted]X)(-[::20,1.75,,,dotted]Y)(-[::90,1.75,,,dotted]Z)%
        }
        \arrow{0}
        \chemfig{%
            N%
            (-[::180,1.25]\clr{H}{1})(-[::-60,1.25]\clr{H}{2})(-[::60,1.25]\clr{H}{3})%
            (-[::0,1.75,,,dotted]X)(-[::90,1.75,,,dotted]Y)(-[::-90,1.75,,,dotted])%
        }
    \schemestop
\end{figure}
The \porb{z} is unchanged by any of the operations of the group i.e. transforms as $\mathup A_1$ and is totally symmetric.
In contrast, $p_x$ and \porb{y}s are neither symmetric nor antisymmetric with respect to the \symop[3]C or \symop[v]{\sigma} operations.
Rather, they go into linear combinations of one another and must therefore be considered together as components of a two-dimensional representation.

The matrices in this \irrep{} will therefore be $2 \times 2$ instead of $1 \times 1$.
Consequently, the character of the identity operation will be $2$ i.e. $\charop{E} = 2$.
A rotation through an angle $\frac{2\pi}{n}$ can be represented by the following transformation: \[\begin{bNiceMatrix}
    x' \\ y'
\end{bNiceMatrix} = \begin{bNiceMatrix}[cell-space-limits = 1pt]
    \cos \left(\frac{2\pi}{n}\right) & \sin \left(\frac{2\pi}{n}\right) \\
    -\sin \left(\frac{2\pi}{n}\right) & \cos \left(\frac{2\pi}{n}\right)
\end{bNiceMatrix} \begin{bNiceMatrix}
    x \\ y
\end{bNiceMatrix}.\]

\noindent The trace for the \symop[n]C rotation matrix is therefore $2 \cos \left(\frac{2\pi}{n}\right)$, which for $n=3$ is $-1$ i.e. \charop{\symop[3]C} = $-1$.
The character for reflection through a plane can be determined by the effect of of reflection through any one of the three planes, since they are all in the same \keyword*{class}.
The easiest operation to use is the reflection through the $XZ$ plane---at which one of the \ce{N-H} bonds lies---which results in $p_x \to p_x$, $p_y \to -p_y$ and $p_z \to p_z$, yielding a trace of $0$ i.e. \charop{\symop[v]{\sigma}} = $0$.
The transformation properties of the $p_x$ and \porb{y}s are thus represented as:
\begin{equation*}
    \begin{NiceArray}
        {r c r c}[hvlines]
        & \symop E & 2 \symop[3]C & 3 \symop[v]{\sigma} \\
        (x, y) & 2 & -1 & 0
    \end{NiceArray}
\end{equation*}
\noindent which is the \symop E \irrep{}.
The $p_x$ and \porb{y}s are degenerate in \PG{C}{3v} symmetry and are taken \emph{together} to form a basis for the two-dimensional \irrep: $\mathup E$.
Treating rotations and binary products as before, we can thus represent the \PG{C}{3v} point group as follows:
\begin{equation*}
    \begin{NiceArray}{c *{3}{r} || c || c}[hvlines]
        \PG{C}{3v} & \symop E & 2 \symop[3]C & 3 \symop[v]{\sigma}\\
        \mathup A_1 & 1 & 1 & 1 & z & x^2 + y^2;\,z^2 \\
        \mathup A_2 & 1 & 1 & -1 & \mathup R_z\\
        \mathup E & 2 & -1 & 0 & (x, y);\;(\mathup R_x, \mathup R_y) & (x^2 - y^2, xy);\,(xz, yz)
    \end{NiceArray}
\end{equation*}
\noindent We can therefore deduce that the $xy$ and $x^2 - y^2$ orbitals are also degenerate, as are $xz$ and $yz$ orbitals.

\begin{problem}
    What are the dimensions of the \irrep{}s of a group with the following classes?
    \tcblower
    $\mathup E,\, \mathup R_1,\, 2 \mathup R_2,\, 2 \mathup R_3,\, 2 \mathup R_4,\, 2 \mathup R_5,\, 5 \mathup R_6,\, 5 \mathup R_7$
\end{problem}

\begin{problem}
    Generate the character table for \PG{D}{4} point group, given $\left\{\PG{D}{4} \middle| \symop E, 2\symop[4]C, \symop[2]C, 2\symop[2]{C'}, 2\symop[2]{C''}\right\}$
\end{problem}

\begin{problem}
    By convention, the $z$-axis is the principle symmetry axis. However, for planar molecules, it is also common to define the $z$-axis as the axis \emph{perpendicular} to the plane. Determine the \irrep{}s to which the metal \dorb{}s belong for \textit{cis}-\ce{PtCl2Br2} (\PG{C}{2v}) using the latter convention, where $z$-axis is perpendicular to the molecular plane, $x$-axis bisects the \ce{Cl-Pt-Cl} bond and $y$-axis bisects the \ce{Cl-Pt-Br} bonds.
\end{problem}